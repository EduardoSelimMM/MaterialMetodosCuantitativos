\documentclass[12pts]{extarticle}
\usepackage[utf8]{inputenc}
\usepackage[margin = 1in]{geometry}
\usepackage[spanish]{babel} 
\decimalpoint
\usepackage{amsmath}
\usepackage[all]{xy}
\usepackage{mathtools}
\usepackage{amsfonts} % mathbb
\usepackage{mathrsfs} % mathscr
\usepackage{enumitem} % letras en enumerate
\usepackage[bottom]{footmisc}
\usepackage{amssymb}
\usepackage{float}
\usepackage{graphicx}
\graphicspath{{imag/}}
\setlength\parindent{0pt} % Quita indents de todo el documento
\author{}
\date{}
\title{Opciones finales}
\begin{document}
\maketitle 
\section{Opciones Rebate} 
\begin{itemize} 
\item Son primas/hermanas de las opciones barrera.
\item Una opción rebote paga una cantidad fija si la barrera se alcanza en cualquier tiempo durante el plazo de la opción, i.e., es del tipo "in".
\item El pago se puede hacer inmediatamente despupes de que toca la barrera o bien hasta el vencimiento de la opción. (Esta ultima situación se conoce como rebate diferido")
\item Hay 2 tipos de opciones rebate:
\begin{enumerate}
\item Uo rebates: Paga dividendos sise tocó la barrera de abajo hacia arriba, i.e., $\exists t \in (0, T]$ tal que $S_t \leq B$. Donde B es la barrera
\item Down rebates: Paga si se tocó la barerra de arriba hacia abajo, i.e., $\exists t \in (0, T]$ tal que $S_t \geq B$. Donde B es la barrera.
\end{enumerate}
\item A estas opciones rebate también se les conoce como opciones \textbf{cash-or-nothing barrera}.
\end{itemize} 
\section{Opciones Exchange}
\begin{itemize}
\item En una opción exhange Europea el payoff es $$(S_T-KQ_T)_+$$ $$=max\{S_t-KQ_T,0\}$$ donde $S_T$ es el precio de la acción al tiempo T. \\ $Q_T$ es el precio de la acción Q al tiempo T. $$(S_T-KQ_T)_+ > 0 \Leftrightarrow S_T-KQ_T > 0 \Leftrightarrow S_T > KQ_T$$ 
\end{itemize}
En una opción exchange se tiene la posibilidad de intercambiar 1 unidad de la acción S por K unidades de la acción K. \\
¿Cuándo convendrá hacer el intercambio?\\ Cuando $S_T > KQ_T$ \\ En caso contrario, i.e., $S_T\leq KQ_T$, NO ejercemos la opción y nos quedamos con la acción S. 



















\end{document}