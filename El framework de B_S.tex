\documentclass[12pts]{extarticle}
\usepackage[utf8]{inputenc}
\usepackage[margin = 1in]{geometry}
\usepackage[spanish]{babel} 
\decimalpoint
\usepackage{amsmath}
\usepackage{mathtools}
\usepackage{amsfonts} % mathbb
\usepackage{mathrsfs} % mathscr
\usepackage{enumitem} % letras en enumerate
\usepackage[bottom]{footmisc}
\usepackage{amssymb}
\usepackage{float}
\usepackage{graphicx}
\graphicspath{{imag/}}
\setlength\parindent{0pt} % Quita indents de todo el documento

\title{El Framework de Black \& Scholes}
\author{}
\date{}
\begin{document}
\maketitle
Cuando se habla del framework de Black \& Scholes, se está haciendo referencia a lo siguiente: 
\begin{enumerate}
\item Para la ditribución del precio de la acción: 
\begin{itemize}
\item El precio al tiempo t del activo subyacente $S_t$ sigue una distribución $log-normal$, es decir $S_t \sim LogN(*,*)$ .
\item El activo subyacente no paga dividendos o paga dividendos a una tasa continua a un nivel proporcional a su precio. 
\end{itemize} 
\item Para el entorno economico:
\begin{itemize}
\item La tasa de interes libre de riesgo con composición continua es \textbf{r} y es constante. Además, se puede prestar y pedir prestado a la misma tasa \textbf{r} 
\item No hay costos de transacción ni impuestos. 
\item Esposible comprar o vender en corto cualquier numero de unidades del activo subyacente. 
\item \textbf{No hay oportunidades de arbitraje}
\end{itemize} 
\end{enumerate}
\textbf{Notación}: Para $t_1 < t_2,\,  R(t_1, t_2) :=log(\frac{S_{t_{2}}}{S_{t_{1}}})$ es la tasa de rendimiento con composición continua, no anualizada de la acción S de $t_1$ a $t_2$, i.e., $$ S_{t_{2}} = S_{t_{1}}\cdot e^{R(t_1, t_2)}$$
¿Qué significa que un activo pague dividendos a una continua? \\ Significa que una unidad de subyacente hoy, se convertirá en $e^{\delta t}$ unidades de acción despues de t periodos. 
$$ 1 \rightarrow e^{\delta t}$$ 
¿En términos de dinero esto que significa? $$ 1\cdot S_0 \rightarrow e^{\delta t}S_t$$ 
A $\delta$ se le conoce como \textit{tasa de dividendos} \\ \\
Se usan rendimientos no anualizados, ya que los rendimientos con composición contunia son aditivos: 
$$ log(\frac{S_{t_{2}}}{S_{t_{1}}})=log(\frac{S_{t_{2}}}{S_{t_{1}}}\cdot \frac{S_{t_{1}}}{S_{t_{0}}})=log(\frac{S_{t_{2}}}{S_{t_{1}}})+log(\frac{S_{t_{1}}}{S_{t_{0}}})$$
Es decir, $R(t_0, t_2)=R(t_0, t_1) + R(t_1, t_2)$ ,  $t_0<t_1<t_2$.\\
Inductivamente para $T \in \mathbb{N_+}$, $$ R(0, T)=R(0,1)+R(1,2)+...+R(T-1,T)$$
Las suposiciones con respecto a la distribución del precio de la ación se puede dar en términos de los rendimientos correspondientes: 
\begin{itemize}
\item Los rendimientos en dos periodos disjuntos son independientes 
\item La media y la varianza de $R(t_1, t_2)$ son proporcionales a $t_2-t_1$
\item En particular , $R(t_1, t_2) \sim N((\alpha-\delta-\frac{1}{2}\sigma^2), \sigma^2 (t_2-t_1))$ 
\end{itemize}
\textbf{IMPORTANTE:}
Por el momento esta es una suposición artificial, más adelante se verá como se llegó a ésta. \\ \\
Entonces, la suposición anterior se puede reescribir como 
$$R(t_1, t_2)=( \alpha-\delta-\frac{1}{2}(t_2-t_1)+\sigma\sqrt{t_2-t_1}Z, \, con Z\sim N(0,1)$$
Donde, 
\begin{itemize}
\item $\alpha$ es la tasa de rendimiento anual esperado de la acción. 
\item $\delta$ es la tasa de dividendo continua 
\item $\sigma$ es la volatlidad de la acción. 
\end{itemize} 
Notemos que $e^{R(t_1, t_2)}$ es el factor de acumulación de $t_1$ a $t_2$ \textbf{estocastico}
\\ \\ 
\textbf{Proposición:}
\begin{itemize}
\item $\mathbb{E}(e^{R(t_1, t_2)})=e^{(\alpha - \delta)(t_2-t_1)}$
\item $Var(e^{R(t_1, t_2)})= e^{2( \alpha -\delta)(t_2 - t_1)} (e^{\sigma^2 (t_2 - t_1)} -1)$
\end{itemize}
\textbf{Demostración:} 
%%%%pendiente 
\\ \\ \\ \\ \\ \\ \\ \\ \\ \\ 
\textbf{Proposición:} $Corr(R(0, t_1), R(0, t_2))=\sqrt{\frac{t_1}{t_2}}$ \\
\textbf{Demostración:}
%%%pendiente 
\\ \\ \\ \\ \\ \\ \\ \\ \\ \\
Sea $S_t$ el precio de la acción al tiempo t. Ya se dijo que $$S_t=S_0 e^{R(0,t)}$$ 
Nótese que $S_t$ tiene distribución $log-normal$ pues $log(S_t)=log(S_0)+R(0,t)$ tiene distribución $log-normal$. 
$$S_t \sim logN(log(S_0)+(\alpha-\delta -\frac{\sigma^2}{2} t), \sigma^2 t)$$ 
$$S_t =S_0 e^{ (\alpha -\delta -\frac{\sigma^2}{2})t +\sigma \sqrt{t}Z}$$

\textbf{Proposición:}
\begin{itemize}
\item $\mathbb{E}(S_t)=S_0 e^{(\alpha-\delta)t}$
\item Sea $x_p$ el p-cuantil de $S_t$, i.e., $\mathbb{P}(S_t \leq x_p)=p$. Entonces $x_p = S_0 \cdot exp\{(\alpha-\delta -\frac{\sigma^2}{2})t +\sigma\sqrt{t}Z_p\}$ donde, $Z_p$ es el p-cuantil de una v.a normal estandar. 
\item $Med(S_t)=S_0 e^{(\alpha-\delta-\frac{\sigma^2}{2})t}$
\item $\mathbb{E}(S_t^k)=exp\{(\alpha-\delta-\frac{\sigma^2}{2})tk + \frac{t\sigma^2  k^2}{2}\}$
\item $\mathbb{E}(S_t^2)=exp\{(\alpha-\delta)2t +\sigma^2 t\}$
\item $Var(S_t)=\mathbb{E}^2 (S_t)(e^{\sigma^2 t}-1)$
 \end{itemize}
\textbf{Demostración:}
%%%% En clase se haceny algunos van  de tarea
\\ \\ \\ \\ \\ \\ \\ \\ \\ \\ 
\textbf{Proposición:} Para $t_1 \leq t_2$
\begin{itemize}
\item $Cov(S_{t_{1}}, S_{t_{2}})=S_0 e^{(\alpha-\delta)(t_1+t_2)}(e^{t_1  \sigma^2 -1})$
\item $Corr(S_{t_{1}}, S_{t_{2}})=\sqrt{\frac{e^{\sigma^2 t_1}-1}{e^{\sigma^2 t_2}-1}}$
\end{itemize} 
\textbf{Deostración:}
%%%%% En clase
\\ \\ \\ \\ \\ 
\textbf{Lema:} Si $Y \sim logN(\mu, \sigma^2)$ entonces: 
\begin{itemize}
\item $\mathbb{E}(Y \cdot \mathbb{1}_{(y>k)})=\mathbb{E}(Y)\Phi(x_0)$
\item  $\mathbb{E}(Y \cdot \mathbb{1}_{(y<k)})=\mathbb{E}(Y)\Phi(-x_0)$
 \end{itemize}
Donde $x_0 := \frac{log(\frac{\mathbb{E}(y)}{k} + \frac{\sigma^2}{2})}{\sigma}$
\\ \textbf{Demostración:}
%%%%%en clase 
\\ \\ \\ \\ \\ 
\textbf{Observación}: En la expresión para $x_0$, no depende de $\mu$ 
\textbf{Proposición:}
\begin{itemize}
\item $\mathbb{E}(S_t \mathbb{1}_{(S_t > k)})=\mathbb{E}(S_t)\Phi(\dot d_1)$.
\item$\mathbb{E}((S_t \mathbb{1}_{(S_t < k)})=\mathbb{E}(S_t)\Phi(-\dot d_1)$
\end{itemize}
Donde, $\dot d_1:= \frac{log(\frac{S_0}{k})+(\alpha -\delta-\frac{\sigma^2}{2})t}{\sigma\sqrt{t}}$ \\
\textbf{Demostración:} 
%%%%%%% pendiente
\\ \\ \\ \\ \\ \\ \\ \\ \\
\textbf{Recordatorio:} Si A es un evento tal que $\mathbb{P}(A)>0$ entonces, $$ \mathbb{E}(X|A)=\frac{\mathbb{E}(X \mathbb{1}_A)}{\mathbb{P}(A)}$$
\textbf{Proposición:} Si $\dot d_2 := \frac{log(\frac{S_0}{k}) + (\alpha -\delta -\frac{sigma^2}{2})t}{\sigma \sqrt{t}}$
\begin{itemize}
\item $\mathbb{P}(S_t >k)=\Phi (\dot d_2)$
\item $\mathbb{P}(S_t <k)=\Phi (-\dot d_2)$
\item $\mathbb{E}(S_t|S_t >k)= S_0 e^{(\alpha-\delta)t} \frac{\Phi (\dot d_1)}{\Phi (\dot d_2)}$
\item $\mathbb{E}(S_t|S_t <k)= S_0 e^{(\alpha-\delta)t} \frac{\Phi (-\dot d_1)}{\Phi (-\dot d_2)}$
\end{itemize}
\textbf{Observación:} $\dot d_2 = \dot d_1 -\sigma \sqrt{t}$\\
\textbf{Proposición:} Un intervalo del $100(1-\gamma)\%$ de confianza/predicción para $S_t$ esta dado por, $$S_0 \cdot exp\{(\alpha-\delta-\frac{\sigma^2}{2})t + Z_{\frac{\gamma}{2}} \sigma \sqrt{t}\}$$


 
\end{document}