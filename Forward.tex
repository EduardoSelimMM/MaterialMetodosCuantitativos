\documentclass[12pts]{extarticle}
\usepackage[utf8]{inputenc}
\usepackage[margin = 1in]{geometry}
\usepackage[spanish]{babel} 
\decimalpoint
\usepackage{amsmath}
\usepackage[all]{xy}
\usepackage{mathtools}
\usepackage{amsfonts} % mathbb
\usepackage{mathrsfs} % mathscr
\usepackage{enumitem} % letras en enumerate
\usepackage[bottom]{footmisc}
\usepackage{amssymb}
\usepackage{float}
\usepackage{graphicx}
\graphicspath{{imag/}}
\setlength\parindent{0pt} % Quita indents de todo el documento
\author{}
\date{}
\title{Forward}
\begin{document}
\maketitle 
¿Cuánto cuesta un Forward? \\ Supongamos que el plazo del Forward es T, el precio pactado es K y la tasa libre de riesgo con composición continua es r. 
\begin{itemize}
\item \textbf{Caso 1:} La acción subyacente, NO paga dividendos. \\ Consideremos 2 potafolios: 
$$ \pi^{(1)}: Una \, accion \, del  \, subyacente $$ 
$$ \pi^{(2)}: Un \, forward \, large \, y \, una \, inversion \, a \, la \, tasa \, libre \, de \, riesgo \, de \, Ke^{-rT}$$
En T, ¿cuánto vale $\pi^{(1)}$ ? 
$$\pi_T^{(1)}=S_T$$
En T, ¿cuánto vale $\pi^{(2)}$? 
$$\pi_T^{(2)} =S_T-K +(Ke^{-rT})e^{rT}$$
$$=S_T-K+K$$
$$=S_T$$
¡Wow! Tengo 2 portafolios que en T valen lo mismo. \\ Por ley del precio unico $$\pi_0^{(1)}=\pi_0^{(2)} .... :)$$ 
¿Cuánto vale hoy el portafolio 1?
$$\pi_0^{(1)}=S_0$$

¿Cuánto vale hoy el portafolio 2?
$$\pi_0^{(2)}= f + Ke^{-rT}$$
  
 Entonces, por :) tenemos que 
$$f+Ke^{-rT}=S_0$$
$$f=S_0-Ke^{-rT}$$
Este es el precio de un forward en el caso de una acción que no paga dividendos. 

\item \textbf{Caso 2:} La acción subyacente paga dividendos continuos a la tasa $\delta$. \\
1 unidad de acción hoy $\rightarrow$ $e^{\delta T}$ unidades de acción en T. 
\\ Consideremos 2 portafolios 
$$ \pi^{(1)}: Una \, unidad \, de \, accion $$ 
$$ \pi^{(2)}: e^{\delta T} \, forwards \, largos\, sobre \, la \, accion  \, y \, una \, inversion \, a \, la \, tasa \, libre \, de \, riesgo \, de \, Ke^{-(r-\delta)T}$$
En T, ¿cuánto vale el portafolio 1 ? 
$$\pi_T^{(1)}=e^{\delta T} S_T$$
En T, ¿cuánto vale el portafolio 2? 
$$\pi_T^{(2)} =e^{\delta T}(S_T-K) +(Ke^{-(r-\delta)T})e^{rT}$$
$$=e^{\delta T}(S_T-K)+Ke^{\delta T}$$
$$=e^{\delta T}S_T-e^{\delta T}K+ke^{\delta T}$$
$$= e^{\delta T}S_T$$
¡Wow! Tengo 2 portafolios que al tiempo T tienen el mismo valor. 
$$\pi_T^{(1)}=\pi_T^{(2)}=e^{\delta T}S_T$$
Por ley del precio único
$$\pi_0^{(1)}=\pi_0^{(2)}... :S$$
¿Cuánto vale hoy el portafolio 1?
$$\pi_0^{(1)}=S_0$$

¿Cuánto vale hoy el portafolio 2?
$$\pi_0^{(2)}= e^{\delta T}f + Ke^{-(r-\delta)T}$$
  
 Entonces, por :S tenemos que 
$$e^{\delta T}f+Ke^{-(r-\delta)T}=S_0$$
Despejando f,
$$S_0-Ke^{-(r-\delta)T}=e^{\delta T}f$$
$$f=S_0e^{-\delta T}-Ke^{-rT}$$
Esto es, el preco del forward en el caso de una acción que paga dividendo continuos. 

\item \textbf{Caso 3:} La acción subyacente paga dividendos discretos.\\
¿Qué significa que la acción pague dividendos dicretos? \\
Significa que el poseedor de la acción recibirá un $Div_j$ al tiempo $t_j$ 
%%va una linea de tiempo %%
\\
Se supondrá que las fichas de pagos de dividendos son conocidas y también el monto de cada dividendo. \\
Considerando 2 portafolios.
$$ \pi^{(1)}: Una \, unidad \, de \, accion $$ 
$$ \pi^{(2)}: Un \, forward \, largos\, sobre \, la \, accion  \, y \, una \, inversion \, a \, la \, tasa \, libre \, de \, riesgo \, de \, Ke^{-rT} \, y \, un \, bono\, libre \, de \, riesgo \, que \, pague \, el \, cupon \, Div_j \, al \, tiempo \, t_j$$
¿Cuánto vale el portafolio 1 al tiempo T? 
$$\pi_T^{(1)}=S_T + \sum_{j=1}^{m} Div_j e^{r(T-t_j)}$$
En T, ¿cuánto vale el portafolio 2? 
$$\pi_T^{(2)} =S_T-K +(ke^{-rT})e^{rT} +\sum_{j=1}^{m} Div_j e^{r(T-t_j)}$$
$$=S_T+\sum_{j=1}^{m} Div_j e^{r(T-t_j)}$$

¡Wow! Tengo 2 portafolios que al tiempo T coincien en el  valor. \\
Por ley del precio único
$$\pi_0^{(1)}=\pi_0^{(2)}... :V$$
¿Cuánto vale hoy el portafolio 1?
$$\pi_0^{(1)}=S_0$$

¿Cuánto vale hoy el portafolio 2?
$$\pi_0^{(2)}=f+K e^{-r T} + \sum_{j=1}^{m} Div_j e^{-rt_j}$$
  
 Entonces, por :V tenemos que 
$$S_0=f+K e^{-r T} + \sum_{j=1}^{m} Div_j e^{-rt_j}$$
Despejando f,
$$f=S_0-ke^{-rT}- \sum_{j=1}^{m} Div_j e^{-rt_j}$$
Precio de un forward para el caso de una acción que paga dividendos discretos. 
\end{itemize} 

¿Cuánto vale el \textbf{precio forward}? \\
El precio forward es el valor de K que hace que f sea 0.
\begin{itemize}
\item Caso 1: $f=S_0-ke^{-rT}$
$$0=S_0-ke^{-rT}$$
$$K=S_0e^{rT}$$ 
\textbf{Notación:} $F_{0,T}(S)$
\item Caso 2:  $f=S_0e^{-\delta T}-ke^{-rT}$
$$0=S_0e^{-\delta T}-ke^{-rT}$$
$$K=S_0e^{(r-\delta)T}$$ 
\textbf{Notación:} $F_{0,T}(S_0)$
\item Caso 3: Tareita

\end{itemize} 
\textbf{Recordatorio} \\
%%Viene un corchete con 3 niveles que no sé hacer :C %%%
\\ \\ 
\textbf{Notación:}
%%corchete grande%%
\\ \\ 
\textbf{Definición:} $F_{0,T}^P (S):=F_{0,T}(S)e^{-rT}$ \\
A $F_{0,T}^P (S)$ se le conoce como prepaid forward price. \\
A $F_{0,T}(S)$ se le conoce como forward price o precio forward. \\
Claramente, 
%%%corchete%%%%%
\\ \\ 
 
\textbf{Recordatorio:} Paridad put-call. 
$$Call-Put=Forward$$
Ya podemos dar 3 versiones de esta paridad. 
 \begin{itemize}
\item Acción no paga dividendos. 
$$C(S,K,T)-P(S,K,T)=S_0-Ke^{-rT} ... (<3)$$
$$=F_{0,T}^{P} (S)-Ke^{-rT}$$
\item Acción que paga dividendos continuos
$$C(S,K,T)-P(S,K,T)=S_0e^{-\delta T}-Ke^{-rT}$$
$$=F_{0,T}^{P} (S)-Ke^{-rT}$$
\item Acción que paga dividendos discretos.
$$C(S,K,T)-P(S,K,T)=S_0-\sum_{j=1}^{m} Div_j e^{-rt_j}-Ke^{-rT}$$
$$=F_{0,T}^{P} (S)-Ke^{-rT}$$
\end{itemize}
Podemos reescribir la ec. $(<3)$
$$C(S,K,T)-P(S,K,T)=S_0-Ke^{-rT}$$
$$C(S,K,T)+Ke^{-rT}=P(S,K,T)=S_0$$
Si yo compro una call, más me vale tener el dinero para poder ejercerla, i.e., necesito tener K en T si decido ejercerla. Si yo compro un aput (estoy comprando el derecho a vender) más me vale tener algo que vender (la acción) en casi de que decida hacerlo. 
\newpage 
\textbf{Definición: Opcion chooser} \\
Una opción chooser con un plazo de T es una opción que tiene la caracteristica de que al tiempo $t_0<T$ ($t_0$ es fijo y contractualmente) el poseedor de ésta puede elegir si la opción es una put o una call con vencimiento en T. 
%%va una linea de tiempo%%%
\\
¿Cuánto cuesta en 0 una opción chooser?
\\
¿Cuánto vale en $t_0$ una opción chooser? 
$$max\{C(S_{t_0},K,T-t_0), P(S_{t_0}, K, T-t_0)\}$$ 

\textbf{Observación:} Usando la paridad put-call $C-P=f, P=C-f$
$$max\{C(S_{t_0},K,T-t_0), P(S_{t_0}, K, T-t_0)\}$$
$$=max\{C(S_{t_0},K,T-t_0), C(S_{t_0},K,T-t_0)-F_{0,T-t_0}^{P} (S)+ke^{-r(T-t_0)}\}$$
\textbf{Recordatorio:} \\
Para $a,b,c \in \mathbb{R}_+$
$$max\{a,b\}=c+max\{a-c,b-c\}$$ 
Entonces, 
$$=C(S_{t_0},K,T-t_0)+max\{0, ke^{-r(T-t_0)}-F_{0, T-t_0}^{P}(S)\}$$
Para una acción que no paga dividendos
$$=C(S_{t_0},K,T-t_0)+max\{0, ke^{-r(T-t_0)}-S_{t_0}\}$$
Así pues, 
$$V_0 = C(S_0,K,T)+ P(S_{0}, Ke^{-r(T-t_0)}, t_0)$$

%%van algunos diagramas (?) %%%%

Considere un portafolio que contiene lo siguiente.
\begin{enumerate}
\item Una call con strike K y plazo T 
\item Una put con strike $ke^{-r(T-t_0)}$ y plazo $t_0$ 
\end{enumerate}
¿Cuánto vale este portafolio al tiempo $t_0$?
\begin{enumerate}
\item Simplemete $C(S_{t_0}, K, T-t_0)$
\item El payoff $(Ke^{-r(T-t_0)}-S_{t_0})$
\end{enumerate}
Por lo tanto, podemos concluir que el precio de la chooser option al tiempo 0 es,
$$V_0=C(S_0,K,T)+P(S_0, Ke^{-r(T-t_0)}, t_0)$$
Este valor también lo pudimos obtener de la siguiente forma. \\
Al tiempo $t_0$, dijimso que el valor de esta opción es 
$$max\{C(S_{t_0}, K, T-t_0), P(S_{t_0},K,T-t_0)\}$$ 
$$=P(S_{t_0},K,T-t_0)+max\{C(S_{t_0},K,T-t_0)-P(S_{t_0},K,T-t_0),0 \}$$
$$=P(S_{t_0},K,T-t_0)+max\{F_{T-t_0}(S)-Ke^{-r(T-t_0)},0 \}$$
$$=P(S_{t_0},K,T-t_0)+max\{S_{t_0}-Ke^{-r(T-t_0)},0 \}$$
$$=P(S_{t_0},K,T-t_0)+(S_{t_0}-Ke^{-r(T-t_0)})_+$$
Con el mismo razonamiento que se utilizo anteriormente 
$$V_0= P(S_0,K,T)+C(S_0, Ke^{-r(T-t_0)}, t_0)$$

Las expresiones anteriores son para el valor de la opción chooser. \\ 
En todo lo que hemos visto esta semana, aparecen $C(*)$ y $P(*)$, i.e., las primas de una call y una put. ¡Necesitamos saber cuanto valen estas! 
\\ 
Afortunada o desafortunadamente, para encontrarlas necesitamos alguna hpt con respecto a la distribución de $S_t$ 
\begin{itemize}
\item Framework B\&S.
\item árboles binomiales.
\end{itemize} 












\end{document} 