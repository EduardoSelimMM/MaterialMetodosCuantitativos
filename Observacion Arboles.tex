\documentclass[12pts]{extarticle}
\usepackage[utf8]{inputenc}
\usepackage[margin = 1in]{geometry}
\usepackage[spanish]{babel} 
\decimalpoint
\usepackage{amsmath}
\usepackage[all]{xy}
\usepackage{mathtools}
\usepackage{amsfonts} % mathbb
\usepackage{mathrsfs} % mathscr
\usepackage{enumitem} % letras en enumerate
\usepackage[bottom]{footmisc}
\usepackage{amssymb}
\usepackage{float}
\usepackage{graphicx}
\graphicspath{{imag/}}
\setlength\parindent{0pt} % Quita indents de todo el documento
\author{}
\date{}
\title{Observación árboles}
\begin{document}
\maketitle
Un comentario sobre árboles (volatilidad). 
\begin{itemize}
\item Hata el momento se han escogido \textbf{u} y \textbf{d} arbitrariamente. Sin embargo NO son arbitrarios. 
\begin{itemize}
\item No se puede hacer a u demasiado bajo.
\item No se puede hacer a d demasiado bajo.
\end{itemize}
\item De hecho, debe ocurrir que $S_u > S_0 e^{(r-\delta)h}$ y $S_d < S_0 e^{(r-\delta)h}$ , i.e., $S_d < S_0 e^{(r- \delta)h} < S_u $. Ya que si ocurriera lo contrario, habría oportunidades de arbitraje. 
\item En particular, en un árbol multiplicativo $d < S_0 e^{(r-\delta)h} < u $ \\ Mientras más alejadas estén u y d, más grande será la varianza del precio de la acción. 
\item Obsérvese que $S_0 e^{(r-\delta)h}$ es el precio forward de la acción, así que debe cumplir que $$ S_d < F_{0,h}(S) < S_u$$ Y en un árbol multiplicativo $$ dS_0 < F_{0,h}(S) <S_0$$ $$d<\frac{F_{0,h}(S)}{S_0}<u... :)$$ 
\item Una primera propuesta para garantizar que u y d satisfagan esta propiedad e hace 
$$u:= \frac{F_{0,h}(S)}{S_0}e^{\sigma \sqrt{h}}$$
$$d:=\frac{F_{0,h}(S)}{S_0}e^{-\sigma \sqrt{h}}$$
Y como $F_{0,h}(S)=e^{(r-\delta)h}$, entonces 
$$u=\frac{F_{0,h}(S)}{S_0}e^{\sigma \sqrt{h}} = \frac{S_0 e^{(r-\delta)h}}{S_0}e^{\sigma \sqrt{j}}$$
$$=e^{(r-\delta)h + \sigma \sqrt{h}}$$
$$d=\frac{F_{0,h}(S)}{S_0}e^{-\sigma \sqrt{h}}=\frac{S_0 e^{(r-\delta)h}}{S_0}e^{-\sigma \sqrt{h}}$$
$$=e^{(r-\delta)h-\sigma \sqrt{h}}$$
Es decir, 
$$u=e^{(r-\delta)h + \sigma \sqrt{h}} \, \, \, \, d=e^{(r-\delta)h-\sigma \sqrt{h}}$$
Esto es al árbol Forward o árbol binomial estandar. 
\item En este caso, la probabilidad de riesgo neutro
$$p^{*}=\frac{e^{(r-\delta)h}-d}{u-d}=\frac{e^{(r-\delta)h}-e^{(r-\delta)h-\sigma \sqrt{h}}}{e^{(r-\delta)h + \sigma \sqrt{h}}-e^{(r-\delta)h-\sigma \sqrt{h}}}$$
$$=\frac{1-e^{-\sigma \sqrt{h}}}{e^{\sigma \sqrt{h}}-e^{-\sigma \sqrt{h}}}=\frac{1-w^{-1}}{w-w^{-1}}$$ Con $w:=e^{\sigma \sqrt{h}}$ 
$$=\frac{1-\frac{1}{w}}{w-\frac{1}{w}}=\frac{\frac{w-1}{w}}{\frac{w^2-1}{w}}=\frac{w-1}{w^2-1}=\frac{w-1}{(w-1)(w+1)}$$
$$=\frac{1}{1+w}=\frac{1}{1+e^{\sigma \sqrt{h}}}$$
$$ \therefore p^{*}=\frac{1}{1+e^{\sigma \sqrt{h}}}$$
Esta es la probabilidad de riesgo neutro para un árbol forward.
\end{itemize}
\textbf{Observación} 
$$1-p^{*}=1-\frac{1}{1+e^{\sigma \sqrt{h}}}=\frac{1+e^{\sigma \sqrt{h}}-1}{1+e^{\sigma \sqrt{h}}}=\frac{e^{\sigma \sqrt{h}}}{1+e^{\sigma \sqrt{h}}}$$
$$=\frac{1}{e^{-\sigma \sqrt{h}}}=\frac{1}{1+e^{-\sigma \sqrt{h}}}$$
Otras propuestas son hacer
$$u:=e^{\sigma \sqrt{h}} \, \, \, \, \, \, \, d:=e^{-\sigma \sqrt{h}}$$
Que sería el arbol de Cox-Ross-Rubinstein.\\
O bien,
$$u:=e^{(r-\delta -\frac{1}{2}\sigma^2)h+\sigma \sqrt{h}} \, \, \, \, \, d:=e^{(r-\delta -\frac{1}{2}\sigma^2)h-\sigma \sqrt{h}}$$
Que sería el árbol de Jarrow-Rudd o árbol lognormal. 
\\ \\
\textbf{Tareita:} Verificar que las esfecificaciones de Cox-Ross.Rubinstein y Jarrow-Rudd satisfacen :)













\end{document} 