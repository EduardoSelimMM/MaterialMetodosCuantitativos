\documentclass[12pts]{extarticle}
\usepackage[utf8]{inputenc}
\usepackage[margin = 1in]{geometry}
\usepackage[spanish]{babel} 
\decimalpoint
\usepackage{amsmath}
\usepackage[all]{xy}
\usepackage{mathtools}
\usepackage{amsfonts} % mathbb
\usepackage{mathrsfs} % mathscr
\usepackage{enumitem} % letras en enumerate
\usepackage[bottom]{footmisc}
\usepackage{amssymb}
\usepackage{float}
\usepackage{graphicx}
\graphicspath{{imag/}}
\setlength\parindent{0pt} % Quita indents de todo el documento
\author{}
\date{}
\title{Arboles}
\begin{document}
\maketitle 
\begin{itemize}
\item Ya vimos el marco continuo o de $B\& S$. \\ En este marco, la suposición más contundente es la de la distribución log-normal. 
\item Ahora veremos un marco de valuación en tiempo discreto que permita valuar opciones. 
\end{itemize}
Supongamos que tenemos una call con cubyacente S , Strike K y vencimiento T. \\ 
Supongamos, que al tiempo T, la acción subyacente puede tomar sólo 2 valores \textbf{$S_u$} o \textbf{$S_d$}, con $S_u>S_d$ y $S_u,  S_d$ conocidos. \\

$$\xymatrix{&S_u  \\ S_0 \ar[ru] \ar[rd] & \\ &S_d }$$ 

\textbf{Ejemplo:} 
$$\xymatrix{&75 \\ 50 \ar[ru] \ar[rd] & \\ &15}$$  
\\ ¿Cuál es el precio de la call? 
$$\xymatrix{& &(S_u-K)_+ \, \, Escenario 1 \\ C \ar[ru] \ar[rd] & & \\ & & (S_d-K)_+  \, \, Escenario 2}$$
\\ Vamos a construir un portafolio que replique el comportamiento de la call sin importar el escenario en el que nos encontremos. \\ 
$\pi$ : $\alpha$ unidades del subyacente y una inversión de $\beta$ pero a la tasa libre de riesgo. \\ 
¿Cuánto vale este portafolio hoy? $$\pi_0=\alpha S_0 + \beta$$ 
¿Cuánto vale este portafolio en T? 
$$\pi_T= \Bigg\{ _{\alpha S_d + \beta e^{rT}}^{\alpha S_u + \beta e^{rT}} $$ 
Siqueremos que este portafolio replique el comportamiento de la call tiene que ocurrir: 
$$ \Bigg\{ _{\alpha S_d + \beta e^{rT} = (S_d-k)_+}^{\alpha S_u+\beta e^{rT}=(S_u-K)_+} $$ 
Para que esto ocurra, tenemos que escoger $\alpha$ y $\beta$ de manera adecuada, es decir, $\alpha$ y $\beta$ que resuelvan el sistema de ecuaciones anterior. \\Entonces, para encontrar $\alpha$: 
$$\alpha=\frac{ \begin{matrix} 
                            |(S_u-k)_+ & e^{rT}| \\ |(S_d-K)_+ & e^{rT}|\end{matrix} }{\begin{matrix}
                                                                                                                          |S_u & e^{rT}| \\| S_d & e^{rT} |\end{matrix}}=\frac{e^{rT}((S_u-K)_+ -(S_d-K)_+)}{e^{rT}(S_u-S_d)}$$
$$=\frac{(S_u-K)_+ -(S_d-K)_+}{S_u-S_d}$$ 
Ahora, para encontrar $\beta$: 

$$\beta=\frac{ \begin{matrix} 
                            |S_u & (S_u-K)_+| \\ |S_d & (S_d-K)_+|\end{matrix} }{\begin{matrix}
                                                                                                                          |S_u & e^{rT}| \\| S_d & e^{rT} |\end{matrix}}=\frac{S_u((S_d-K)_+ -S_d(S_u-K)_+}{e^{rT}(S_u-S_d)}$$
Entonces con estos $\alpha \, y \, \beta$ el portafolio replica al payoff de la call. \\
¡WOW! Tengo 2 portafolios que al tiempo T valen lo mismo \\
$\Rightarrow$ Ambos portafolios deben valer hoy lo mismo (Ley del precio) 
$$\pi_0=c \Rightarrow c=\alpha S_0 +\beta$$
$$c=(\frac{(S_u-K)_+ -(S_d-K)_+}{S_u-S_d})S_0+\frac{S_u((S_d-K)_+ -S_d(S_u-K)_+}{e^{rT}(S_u-S_d}$$
Vamos a intentar generalizar esta idea para cualquier derivado (Estilo Europeo).
\\ 
Considere un derivado con subyacente S, donde S es una acción que paga dividendos continuos con una tasa $\delta$. Supongamos al tiempo T, la acción solo puede valer $S_d$ o $S_u$
$$\xymatrix{&S_u  \\ S_0 \ar[ru] \ar[rd] & \\ &S_d }$$ 
                  
Sea $V_0$ el valor del derivado hoy (es lo que estamos buscando), entonces: 
\begin{itemize}
\item $V_u$ es el valor del derivado en caso de que el precio de la acción sea $S_u$
\item $V_d$ es el valor del derivado en caso de que el precio de la acción sea $S_d$
\end{itemize}
$$\xymatrix{&V_u  \\ V_0 \ar[ru] \ar[rd] & \\ &V_d }$$
Como antes, considere el siguiente portafolio: 
\begin{itemize} 
\item $\pi$: $\alpha$ unidades del subyacente y $\beta$ pesos a invertir a la tasa libre de riesgo. 
\end{itemize}
¿Cuánto vale hoy $\pi$?
$$\pi_0=\alpha S_0+\beta$$ 
\textbf{Recordatorio:} Una unidad hoy se vuelve $e^{rT}$ unidades en T. 
\\  \\ 
¿Cuánto vale $\pi$ en T? 
$$\pi= \Bigg \{_{\alpha e^{\delta T} S_d +\beta e^{rT}}^{\alpha e^{ \delta T} S_u +\beta e^{rT}}$$
Si pretendemos que este portafolio replique al derivado tiene que ocurrir que 
$$\Bigg \{_{\alpha e^{\delta T}S_d +\beta e^{rT}=V_d}^{\alpha e^{\delta T}S_d+\beta e^{rT}=V_u}$$

La expresión anterior representa un sistema de ecuaciones, entonces: 
$$\alpha=\frac{ \begin{matrix} 
                            |V_u & e^{rT}| \\ |V_d & e^{rT}|\end{matrix} }{\begin{matrix}
                                                                                                                          | e^{\delta T}S_u & e^{rT}| \\| e^{\delta T}S_d & e^{rT} |\end{matrix}}=\frac{e^{rT}(V_u-V_d}{e^{\delta T}e^{rT}(S_u-S_d)}$$
$$=e^{-\delta T}\frac{V_u-V_d}{S_u-S-d}$$ 

Y para $\beta$ 
$$\beta= \frac{ \begin{matrix} 
                            |e^{\delta T}S_u & V_u| \\ |e^{\delta T}S_d & V_d|\end{matrix} }{e^{\delta T} e^{rT}(S_u-S_d)}=\frac{e^{\delta T}(S_u Vd_d-S_dV_u}{e^{\delta T}e^{rT}(S_u-S_d)}$$
$$=e^{-rT}\frac{S_u V_d-S_d V_u}{S_u-S_d}$$

Así entonces, 
$$\alpha = e^{-\delta T}\frac{V_u-V_d}{S_u-S-d} \,  \, \, \, \beta=e^{-rT}\frac{S_u V_d-S_d V_u}{S_u-S_d}$$
Para que el portafolio replique al derivado hoy debe comprar $\alpha = e^{-\delta T}\frac{V_u-V_d}{S_u-S-d}$ unidades del activo subyacente e invertir $\beta=e^{-rT}\frac{S_u V_d-S_d V_u}{S_u-S_d}$ a la tasa libre de riesgo. \\ \\ 
Entonces, $$ V_0=\pi_0=\alpha S_o + \beta $$ 
$$= e^{-\delta T}\frac{V_u-V_d}{S_u-S_d} S_0 + e^{-rT}\frac{S_u V_d-S_d V_u}{S_u-S_d}$$
Esto es la prima del derivado usando el modelo de árbol. 
\\ \\ 
\textbf{Ejemplo:} Considérese una call con Strike K=55, con plazo de un año, sobre una acción que \textbf{NO} paga dividendos y al final del año puede valer 40 o 60. El precio actual de la acción es de \$50. Si la tasa libre de riesgo es del 5\%, ¿cuánto vale la call? 

$$\xymatrix{&S_u=60 \\ S_0= 50 \ar[ru] \ar[dr] & \\ &S_d=40}.$$
$$\xymatrix{&V_u=(60-K)_+=(60-55)_+ \\ C \ar[ru] \ar[dr]&  \\ & V_d=(40-K)_+=(40-55)_+}$$
\\ \\ Como la acción no pada dividendos, entonces $\delta =0$. Sustituyendo en la expresión anterior 
$$ \alpha = e^{-\delta T}\frac{V_u-V_d}{S_u-S-d}=e^{-0*T}\frac{5-0}{60-40} = \frac{5}{20}=\frac{1}{4}$$ 
$$  \beta=e^{-rT}\frac{S_u V_d-S_d V_u}{S_u-S_d}=e^{-0.05(1)}\frac{60(0)-40(5)}{60-40}=e^{-0.05}\frac{200}{20}=-10e^{-0.05}$$
Recordando que 
$$C=\alpha S_0 + \beta $$
Tenemos entonces que, 
$$C=\frac{1}{4}S_0-10e^{-0.05}$$ 
$$=\frac{1}{4}(50)-10e^{-0.05}$$
$$=2.98$$ 

\textbf{Observación:}
$$ V_0 = e^{-\delta T} \frac{V_u-V_d}{S_u-S_d}S_0 + e^{-rT} \frac{S_uV_d -S_d V_u}{S_u-S_d}$$ 
$$=\frac{e^{-rT}}{S_u-S_d}[e^{(r-\delta)T}(V_u-V_d)S_0 +S_uV_d-S_dV_u]$$
$$=\frac{e^{-rT}}{S_u-S_d}[V_u(S_0 e^{(r-\delta)T}-S_d)+V_d(S_u-e^{(r-\delta)T}S_0)]$$
$$=e^{-rT}[V_u \frac{S_0 e^{(r-\delta)T}-S_d}{S_u-S_d}+V_d(1-\frac{S_0e^{(r-\delta)T}S_d}{S_u-S_d})]$$
$$=e^{-rT}(p^{*}V_u+(1-p^{*})V_d)  \, \, \, se parece a una esperanza$$ 
$$p^{*}=\frac{S_0 e^{(r-\delta)T}-S_d}{S_u-S_d}$$ 
Si logro garantizar que $p^{*} \in (0,1)$ entonces podré escribir 
$$V_0=e^{-rT}\mathbb{E}^{*}(V)$$
Donde V es una v.a., 
$$V=\Bigg \{_{V_d \, \, con \,proba\, 1-p^{*}}^{V_u \, \, con\, proba\, p^{*}}$$
¿Cómo garantizamos que $p^{*}=\frac{S_0 e^{(r-\delta)T}-S_d}{S_u-S_d}$ este entre 0 y 1? 
$$0<\frac{S_o e^{(r-\delta)T}-S_d}{S_u-S_d} \, \, \, y  \frac{S_o e^{(r-\delta)T}-S_d}{S_u-S_d}<1$$
$$\Leftrightarrow$$ 
$$ 0< S_0 e^{(r-\delta)T}-S_d \, \, \, y S_0 e^{(r-\delta)T}-S_d<1$$
$$\Leftrightarrow$$
$$S_d <S_0e^{(r-\delta)T} \, \, \, y \, \, \, S_0e^{(r-\delta)T}<S_u$$
$$\Leftrightarrow$$
$$S_d<S_0e^{(r_\delta)T}<S_u$$
Es decir, 
$$p^{*} \in (0,1) \Leftrightarrow S_d<S_oe^{(r-\delta)T}<S_u$$
\textbf{Proposición:} Si no hay oportunidades de arbitaje, entonces $ S_d<S_oe^{(r-\delta)T}<S_u $ \\ \\ 
\textbf{Demostración:}\\ 
 Por reducción al absurdo, supongamos $S_d \geq S_0 e^{(r-\delta)T} \,\, o \,\,  S_u \leq S_0 e^{(r-\delta)T}$  \\
CASO 1: $S_d \geq S_0 e^{(r-\delta)T}$ 
\\ 
Considérese la siguiente estrategia: Venda en corto hoy la acción y lo obtenido inviertalo a la tasa libre de riesgo. ¿Cuánto desembolasamos hoy? Nada.
¿En T qué compromisos tenemos? Tengo que devolver la acción, i.e., tengo que devolver $S_d e^{\delta T}$ pesos, pero yo tengo $S_0 e^{rT}$ de mi inversión. 
¿Me alcanza para comprar la acción? ¿$S_0e^{(rT} > S_de^{\delta T}$ \\ $\Leftrightarrow$ \\ $S_0e^{(r-\delta)T}>S_d$ se cumple por hip. \\
Es decir, construí una estrategía en la que hoy no desembolso nada y en el futuro tuve una ganancia positiva. $\nabla$ Contradice a la hipotesis de no arbitraje. 
\\ 
CASO 2: $S_u \leq S_0 e^{(r-\delta)T}$ 
\\ Análogo. $\blacksquare$ 
\\ \\ 
\textbf{Conclusión:} Si no hay oportunidad de arbitraje entonces, 
$$S_d<S_0e^{(r-\delta)T}<S_u$$
Entonces, 
$$p^{*}:=\frac{S_0e^{(r-\delta)T}-S_d}{S_u-S_d} \in (0,1)$$
Entonces, puede escribir como una esperanza legitima
$$V_0=e^{-rT}\mathbb{E}^{*}(V)$$ 
$$V=\Bigg \{_{V_d \, \, \, \, 1-p^{*}}^{V_u  \, \, \, \, p^{*}}$$
A $p^{*}$ se le conoce como probabilidad de riesgo neutro.   
\\ \\ \\ 
¿Qué opinan del modelo de árbol hasta el momento?
\begin{itemize}
\item Así como está lleva a una formula muy sencilla de valuación :) 
\item Suponer que en el futuro una acción sólo dos valores puede ser dificil de creer :( 
\end{itemize}

Vamos a intentar arreglar este asunto
$$\xymatrix{\\&&&S_{uuu}\\&&S_{uu} \ar[ru] \ar[dr]&\\ & & & S_{uud}\\&S_u \ar[ru] \ar[dr]&&\\&&&S_{udu}\\ &&S_{ud} \ar[ru] \ar[dr]&\\ &&&S_{udd}\\S_0 \ar[ru] \ar[dr]&&&\\&&&S_{duu}\\ &&S_{du} \ar[ru] \ar[dr]&\\&&&S_{dud}\\&S_d \ar[ru] \ar[dr]&&\\&&&S_{ddu}\\&&S_{dd} \ar[ru] \ar[dr]&\\&&&S_{ddd} }$$
En este caso, se supondrá que la acción puede tomar 8 valores al tiempo T ($2^3=8$)
\\
Haciendo un refinamiento en 3 subpates de este arbol llegamos a 8 posibles valores en T. Para un refinamiento del intervalo $[0,T]$ en n subpartes, ¿cuántos posibles valores puede tomar $S_T$? \textbf{$2^n$}
\\ $2^n$ es un número de posibilidades grande. Es decir, con este refinamiento tenemos $2^n$ escenarios. 
\begin{itemize} 
\item $2^{n}$ es bueno porque es grande.
\item $2^n$ es malo porque es grande 
\end{itemize}
Vamos a considerar árboles no tan grandes. 
%%Va diagrama%% 
A un árbil de este estilo se le conoce como árbol que recombina valores. Para un refinamiento de [o,T] en n subdivisiones de un arbol que recombina, ¿cuántos escenarios finales tendremos? n+1.
\\ ¿Para qué nos servirán estos arboles refinados? 
\\ Para fijar ideas, consideremos un refinamiento en 2 subperiodos. 
%%% Va diagrama %%% 
 \\ \\ \\ Y el correspondiente árbol para el derivado 
%% va diagrama%%
\\ \\ \\ ¿Cuánto vale $V_u$? \\
Entonces, 
$$V_0=\pi_0 =\alpha S_0 +\beta $$
$$=e^{-\delta T} \frac{V_u-V_d}{S_u-s_d}S_0+e^{-rT}\frac{S_uV_d-S_dV_u}{S_u-S_d}$$ 
Esto es la prima del derivado usando el modelo de árbol. 
$$V_u=e^{-\delta \frac{T}{2}}\frac{V_{uu}-V_{ud}}{S_{uu}-S_{ud}}S_u+e^{-r\frac{T}{2}}\frac{S_{uu}V_{ud}-S_{ud}V_{uu}}{S_{uu}-S_{ud}}$$
¿Cuánto vale $V_d$?
$$V_d=e^{-\delta \frac{T}{2}}\frac{V_{ud}-V_{dd}}{S_{ud}-S_{ud}}S_u+e^{-r\frac{T}{2}}\frac{S_{ud}V_{dd}-S_{dd}V_{ud}}{S_{ud}-S_{dd}}$$
¿Cuánto vale $V_0$? 
$$V_0=e^{-\delta \frac{T}{2}}\frac{V_{u}-V_{d}}{S_{u}-S_{d}}S_0+e^{-r\frac{T}{2}}\frac{S_{u}V_{d}-S_{d}V_{u}}{S_{u}-S_{d}}$$

\textbf{Para un periodo} 
$$\xymatrix{&S_u \\ S_0 \ar[ru] \ar[dr] & \\ &S_d}$$
 $$\xymatrix{&V_u \\ V_0 \ar[ru] \ar[dr] & \\ &V_d}$$ 
$$V_0 =\alpha S_0 +\beta = e^{-\delta T}\frac{V_{u}-V_{d}}{S_{u}-S_{d}}S_0+e^{-rT}\frac{S_{u}V_{d}-S_{d}V_{u}}{S_{u}-S_{d}}$$ 
También, 
$$V_0=e^{-rT}[p^{*}V_u+(1-p^{*})V_d]$$
$$p^{*}=\frac{S_0 e^{(r-\delta)T}-S_d}{S_u-S_d} \in (0,1)$$ 
%%%Va un diagrama %%% 
Según lo que vimos para un periodo
$$ V_u=e^{-\delta \frac{T}{2}}\frac{V_{uu}-V_{ud}}{S_{uu}-S_{ud}}S_u+e^{-r\frac{T}{2}}\frac{S_{uu}V_{ud}-S_{ud}V_{uu}}{S_{uu}-S_{ud}}$$
$$V_d=e^{-\delta \frac{T}{2}}\frac{V_{ud}-V_{dd}}{S_{ud}-S_{ud}}S_u+e^{-r\frac{T}{2}}\frac{S_{ud}V_{dd}-S_{dd}V_{ud}}{S_{ud}-S_{dd}}$$
$$V_0=e^{-\delta \frac{T}{2}}\frac{V_{u}-V_{d}}{S_{u}-S_{d}}S_0+e^{-r\frac{T}{2}}\frac{S_{u}V_{d}-S_{d}V_{u}}{S_{u}-S_{d}}$$
%%Va otro diagrama%%
Si quisieramos ponercomo como valores esperados:
$$V_u=e^{-r \frac{T}{2}}[V_{uu}p_1^{*}+V_{ud}(1-p_1^{*})] \, \, p_1^{*}=\frac{S_u e^{r \frac{T}{2}}-S_{ud}}{S_{uu}-S_{dd}}$$
$$V_d=e^{-r \frac{T}{2}}[V_{du}p_2^{*}+V_{dd}(1-p_2^{*})] \, \, p_2^{*}=\frac{S_d e^{r \frac{T}{2}}-S_{dd}}{S_{du}-S_{dd}}$$
$$V_0=e^{-r \frac{T}{2}}[V_{u}p^{*}+V_{d}(1-p^{*})] \, \, p^{*}=\frac{S_0 e^{r \frac{T}{2}}-S_{d}}{S_{u}-S_{d}}$$
\textbf{Ojo:} $p_1^{*} \, , p_2^{*} \, , p^{*}$ son diferentes. 
\\
%%va otro diagrama%% 
Vamos a suponer también: 
\begin{itemize} 
\item $S_u=uS_o$
\item $S_{d}=dS_o$
\item $S_{uu}=uuS_o=u^2S_0$
\item $S_{dd}=ddS_o=d^2S_0$
\item $S_{ud}=S_{du}=udS_0$
\end{itemize} 
Decimos que el árbol es multiplicativo. 
$$V_u= e^{-r \frac{T}{2}}[V_{uu}p_1^{*}+V_{ud}(1-p_1^{*})] \, \, p_1^{*}=\frac{u S_0 e^{r \frac{T}{2}}-S_{ud}}{S_{uu}-S_{dd}} = \frac{u S_0 e^{r \frac{T}{2}}-udS_0}{uuS_0-udS_0}=\frac{e^{r \frac{T}{2}} -d}{u-d} $$  

$$V_d=e^{-r \frac{T}{2}}[V_{du}p_2^{*}+V_{dd}(1-p_2^{*})] \, \, p_2^{*}=\frac{S_d e^{r \frac{T}{2}}-S_{dd}}{S_{du}-S_{dd}}= \frac{dS_0 e^{r \frac{T}{2}}-ddS_0}{udS_0-ddS_0}=\frac{e^{r\frac{T}{2}} -d}{u-d}$$

$$V_0=e^{-r \frac{T}{2}}[V_{u}p^{*}+V_{d}(1-p^{*})] \, \, p^{*}=\frac{S_0 e^{r \frac{T}{2}}-S_{d}}{S_{u}-S_{d}}=\frac{S_0 e^{r \frac{T}{2}}-dS_0}{uS_0-dS_0}=\frac{e^{r\frac{T}{2}} -d}{u-d}$$

\textbf{Moraleja:} Si el árbol NO es multiplicativo, tendremos diferentes $p^{*}$ en cada sub-árbol. Si el árbol sí es multiplicativo, tendremos una única $p^{*}$ 
$$p^{*}= \frac{e^{rh}-d}{u-d}$$
Donde h es la longitud de cada subintervalo, i.e, dividimos a T en n intervalos, cada uno de longitud h.  \\

\textbf{Ejemplo:} Considérese una put Europea con plazo de T=6 meses, sobre una acción que no paga dividendos ($\delta=0$) cuya dinamica estocastica tiene el siguiente comportamiento
%% va diagarama %% 
Si la tasa libre de riesgo es del 6\% anual. Determine la prima de la call. 
\\ 
\textbf{Solución:} Observese que el árbol recombina valores. ¿Es multiplicativo? 
\begin{itemize}
\item $195=u150$
\item $105=d150$
\item $253.5=u^{2}150$
\item $136.5=ud150$
\item $73.5 =d^2 150$
\end{itemize}
$$u=\frac{195}{150}=1.3 \, \, \, , d= \frac{105}{150}=0.7$$ 
\begin{itemize}
\item $150ud=150(1.3)(0.7)=136.5$
\item $150 d^2=150(0.7)^2=73.5$
\item $150 u^2=150(1.3)^2=253.5$
\end{itemize}
$ \therefore $El  árbol  sí  es  multiplicativo 
$$p^{*}=\frac{e^{rh}-d}{u-d}=\frac{e^{0.06 \frac{3}{12}}-0.7}{1.3-07}=0.525188$$
%%%% Van 2 diagramas %%%%
$$V_u=e^{-0.06 \frac{3}{12}}[p^{*}V_{uu}+(1-p{*})V_{du}]=e^{-0.06 \frac{3}{12}}[p^{*}0+(1-p{*})23.5]=10.99195$$
$$V_d=e^{-0.06 \frac{3}{12}}[p^{*}V_{ud}+(1-p{*})V_{dd}]=e^{-0.06 \frac{3}{12}}[p^{*}23.5+(1-p{*})86.5]=52.61791$$
$$V_0=e^{-0.06 \frac{3}{12}}[p^{*}V_{u}+(1-p{*})V_{d}]=e^{-0.06 \frac{3}{12}}[p^{*}10.9195+(1-p{*})52.6179]=30.2985$$

$$\therefore V_0=30.2985$$ 










\end{document}