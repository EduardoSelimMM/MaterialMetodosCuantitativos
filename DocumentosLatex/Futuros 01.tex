\documentclass[12pts]{extarticle}
\usepackage[utf8]{inputenc}
\usepackage[margin = 1in]{geometry}
\usepackage[spanish]{babel} 
\decimalpoint
\usepackage{amsmath}
\usepackage[all]{xy}
\usepackage{mathtools}
\usepackage{amsfonts} % mathbb
\usepackage{mathrsfs} % mathscr
\usepackage{enumitem} % letras en enumerate
\usepackage[bottom]{footmisc}
\usepackage{amssymb}
\usepackage{float}
\usepackage{graphicx}
\graphicspath{{imag/}}
\setlength\parindent{0pt} % Quita indents de todo el documento
\author{}
\date{}
\title{Futuros}
\begin{document}
\maketitle 
\begin{itemize} 
\item Un contrato a futuro es un contrato forward estandarizado \textbf{exchange-traded} (es comprado/vendido en un mercado organizado) 
\item Al vencimiento, el comprador de un contrato de futuros habrá recibido, durante la vida del contrato, el exceso del precio del activo subyacente sobre el forward prive. 
\item Sin embargo, hay diferencias entre un futuro y un forward: 
\begin{itemize}
\item Un contrato de futuros es un contrato estandarizado tradeado en exchanges (mercados formales). Los contratos están disponibles para numeros  especificos de acciones del subyacente, para fechas de expiración especificas \\ $\rightarrow$ Un contrato forward es un contrato "personalizado" entre un comprador y un vendedor. 
\item Como los futuros se tradean en mercadosformales, los futuros son liquidos. Se puede vender un contrato entrando en la posición opuesta. Si se está largo en un contrato que expira en Diciembre, se puede estar corto sobre el mismo contrato que expire en Diciembre, concelando efectivamente el contrato.
\item Los contartos de futuros estan \textbf{marked-to-market}. Esto significa que cada día se acredita el incremento del precio al comprador o el decrecimiento del precio de carga alcomprador. 
\\ $\rightarrow$ Con el fin de llevar a cabo esto, las partes del contrato deben mantener una cuenta de margen (margin account) que es acreditada o cargada. Estas cuentas ganan interés. 
\\ $\rightarrow$ Por tanto, el comprador acumula o paga interés sobre los incrementos o decrementos en el precio. En un contrato fordwar no hay pago de interés sin importar  cuanto del forward price cambia. 
\\ $\rightarrow$ El \textbf{scttlement diario} y la cuenta de margen reducen el riego de crédito para los futuros. 
\end{itemize}

\item Un contarto de futuros popular es el contrato de futuros \textbf{S\&P 500 index}
\\ $\rightarrow$ Cada contrato se basa en un nacional de 250 veces el índice (más adelante profundizaremos más en esto)
\end{itemize}

\textbf{Suposición:} Se supondrá que el future price es el mismo que el forward price. (Esto hará sentido más adelante)
\\ \\ 
\textbf{Ejemplo:} Para una acción:
\begin{itemize}
\item El precio de la acción hoy es de \$40.
\item La acción paga dividendos continuos a una tasa del 4\%
\item La tasa libre de riesgo con composición continua es del 10\%
\end{itemize} 
La parte A compra 10,000 contratos de futuros a la parte B. Cada contrato permite comprar una acción \\
$\rightarrow$ Despues de un día, el precio de la acción se incrementa a \$42. \\
Calcular la cantidad que paga la parte B a la parte A después de un día. 
\\ 
\textbf{Solución:} 
\\ 
Recuerdese que el forward price es $S_0 e^{(r-\delta)T}$, por lo tanto, el precio futuro (precio forward) con este plazo de 6 meses es de 
$$10,000S_0 e^{(r-\delta)\frac{180}{360}} =10,000(40)e^{(0.1-0.04)\frac{180}{360}}=412,012.$$
Esto es lo que la parte A le pagará a la parte B al final de 6 meses a cambio de 10,000 acciones. Como se sabe que el precio de la acción subió a \$42 el día siguiente, el precio futuro (precio forward) es de 
$$10,000S_1 e ^{(r-\delta)\frac{179}{365}}=10,000(42)e^{(0.1-0.04)\frac{179}{365}}=432,542.$$
Eso significa que la parte B paga inmediatamente a la parte A 
$$432,542-412,012=20,530$$

Para garantizar los pagos mark-to-market, cada parte tiene una margin account que se carga o acredita de los scttlement diarios. 
 \\$\rightarrow$ La margin account acumula interés.
 \\$\rightarrow$ Por lo tanto, cada parte gana interés de las "ganancias" del mark-to-market, a diferencia de un contrato forward que sólo paga al final. 
 \\$\rightarrow$ La cantidad inicial del margin es un portencaje del valor nacional
 \\$\rightarrow$ El \textbf{valor nacional} es el valor de los assets subyacentes del contrato. Por ejemplo, un contrato de futuros sobre 2500 acciones del indice S\&P, con el precio del índice igual a 2000 tiene un nacional de $250(2000)=500,000$
 \\$\rightarrow$ El porcentaje del valor nacional que se usa para el margin inicial se determina por el mercado, basándose en la volatilidad del activo subyacente.(Mientras mas volatil más grande será este porcentaje) 
 \\$\rightarrow$ El margin puede decrecer si éste se usa para pagar los marks-to-market. Por lo tanto,hay un requisito de \textbf{mantenimiento del margen.}
 \\$\rightarrow$ El maintenance margin es un porcentaje alto del margin inicial, generalmente $70\%$ u $80\%$ 
 \\$\rightarrow$ Si el margin account está por debajo del maintenance margin, el broker hará un \textbf{margin call} al inversionista. Si el inversionista no aporta fondos para incrementar el margin acoount al nivel del margin inicial, el broker cerrará la posición y devolverá el margin restante al inversionista. 
\\ \\
\textbf{Ejemplo:} Un inversionista compra 100 contrato de futuros, el precio futuro es de 2,300. El margin inicial es del $10\%$ y el maintenance margin es $80\%$ del margin inciial. Se paga una tasa de interés del $50\% $ efectiva anual en el margin account. Los precios-futuros cambian a \$2350 el día 1 y a \$2200 el día 2. Calcular la cantidad en la margin account los días 0,1,2 u si hay algun call el día 2. 
\\ \\ 
\textbf{Solución:}
\\ 
El precio inicial es de $100(2300)=230,000$ (nacional). Por lo tanto el margin inicial es de $230,000(0.10)=23,000$ Y el maintenance margin es de $23,000(0.8)=18,400$ 
\\ 
\begin{itemize}
\item El día 1, la margin account creció a $23,000(1.05)^{\frac{1}{365}}=23,003.07.$ El mark.to-market es $(2350-2300)(100)=5000$, esto hará que se incremente el margin account a $23,003.07+5,000=28,003.07$ .
\item El dia 2, la margin account crecerá a $28,003.07(1.05)^{\frac{2}{365}}=28,006.81$. El mark-to-market es $(2,200-2,350)(100)=-15,000$ Disminuyendo la margin a $28,006.81-15,000=13,006.81 \nabla$  (es menor que el maintenance margin) \\ Como esta cantidad es menor que el maintenance margin (18,400) se hace un margin call de $23,000-13,006.81=9,993.19$  
\end{itemize}

\textbf{Ejemplo:} Un inversionista contrata a un broker el 3 de mayo para comprar 2 futuros de oro que vende el 14 de mayo.
\begin{itemize}
\item El future price el 3 de mayo esde \$350 por onza.  
\item El valor nacional es de 100 onzas por contrato. 
\item El margen inicial es de \$1250 por contrato y la cuenta de margen gana un 3\% anual con composición continua. (El margen \textbf{NO} es el precio para entrar al contrato de futuros.)
\item El maintenance margin es de 75\% del  margin inicial. 
\end{itemize} 
Se sabe que el future price toma los siguientes valores. \\
350, 347, 348.4, 344.1, 342, 343.8, 345.4, 341.2, 341, 340.5, 342.5 \\
\begin{center}
 \begin{tabular}{||c c c c c c||} 
 \hline
 Día & Future Price & Ganancia Diaria & Ganancia acumulada & Cuenta de margen & Margin call \\ [0.5ex] 
 \hline\hline
 & 350 &&2500&\\ 
 \hline
 347& 2*100(397-350)&-600&1900.205&no \\
 \hline
 348.4 & 2*100(348.4-397)& A & 2180.362 & no \\
 \hline
 344.1 & 860 & B & 1320.541 & sí\\
 \hline
 342 & 420 & C & 2080.205 & no\\  
 \hline
343.8 & 360 & D &&\\ 
\hline
345.4 & 320 & & & \\
\hline 
341.2 & 840 & & & \\
\hline 
341 & 40 & &  & \\
\hline 
340 & 200 & & & \\
\hline
345.5 & 500 & & & \\ [1ex]
\hline
\end{tabular}
\end{center}

\begin{itemize} 
\item $2500e^{0.03(\frac{1}{365})}-600=1900.205$
\item  $1900.205e^{0.03(\frac{1}{365})}+280=2180.362$
\item  $2180.362e^{0.03(\frac{1}{365})}-860=1320.541$
\item  $2500e^{0.03(\frac{1}{365})}-420=2080.205$
\item  $-600e^{0.03(\frac{1}{365})}+280=-320.0493$
\item  $-320e^{0.03(\frac{1}{365})}-860=-1180.076$
\item  $-1180.076e^{0.03(\frac{1}{365})}-420=-1600.1726$
\end{itemize}
\textbf{Tareita:} Terminar esta tabla en R. 
\\ \\ 
\textbf{Proposición:} La ganancia total acumulada al tiempo T es $F_T-F_0$ (payoff del contrato de futuros), donde $F_T$ es el future price al tiempo T.

\section{Opciones sobre contratos de futuros} 
\begin{itemize}
\item Una call con strike K sobre un contrato de futuros tiene un payoff de$(F_T-K)_+=max\{0,F_t-K\}$
\item Una put con strike K sobre u contrato de futuros tiene un payoff de $(K-F_T)_+=max\{0,K-F_T\}$ donde $F_T$ es el future price al tiempo T del contrato de futuros.
\end{itemize}
Sean $C(F_0,K,T) \, \, y \, \, P(F_0,K,T)$ los precios al tiempo 0 de una call y una put sobre futuros, respectivamente. \\ \\
\textbf{Proposición:} $(F_T-K)_+ -(K-F_T)_+ = F_T-K$ (Ya lo hemos demostrado) \\
Obserevese que $$(F_T-K)_+-(K-F_t)_+ = F_T-K$$ 
$$=F_T-F_0+F_0-K$$
$$=F_T-F_0+F_0-K$$
Considerese un portafolio que consiste de:
\begin{itemize}
\item Entrar a un contrato de futuros
\item Una inversión a la taa libre de riesgo de $(F_0-K)e^{-rT}$
\end{itemize}
$\rightarrow$ ¿Cuánto vale en 0 este portafolio? $(0+(F_0 -K)e^{-rT})$
$\rightarrow$ ¿Cuánto vale en T este portafolio? $(F_T-F_0)+(F_0-K)$ \\ "call-put=futuros+$(F_0-K)e^{-rT}$"" (Paridad put-call para futuros. 
\\
\section{Valuación de opciones sobre contratos de futuros con arboles}
Los precios de opciones sobre futuros requieren obtenerse de manera diferente ya que a diferencia de las acciones o índices accionarios, no se requiere inversión inicial para entrar en un contrato de futuros (si es que no se consideta el margin requirement)  
\begin{itemize} 
\item Al tiempo 0, el future price es $F_0$ 
\item Al tiempo h, el future price es $F_0(u) \, o \, F_0(d)$ con $u>1 \, y \, d<1$ 
\item Al tiempo h, el payoff de un derivado es $V_u \, o \, V_d$, dependiendo del valor de $F_h$
\end{itemize} 

$$\xymatrix{&F_0(u) \\ F_0 \ar[ru] \ar[dr] & \\ &F_0(d)}$$ 
 $$\xymatrix{&V_u \\ V_0 \ar[ru] \ar[dr] & \\ &V_d } $$
Como antes, se construirá un portafolio que replica. \\
Considérese el siguiente portafolio: 
\begin{itemize} 
\item Entrar a $\alpha$ contratos de futuros
\item Invertir $\beta$ pesos a la tasa libre de riesgo. 
\end{itemize} 
¿Cuánto vale el portafolio al tiempo 0? $\alpha (0)+\beta=\beta$ 
\\ 
Recuérdese que el payoff de un contrato de futuros es el cambio en los future prices, i.e., $F_h -F_0$ \\
¿Cuánto vale ete portafolio al tiempo h? $\alpha (F_h-F_0)+\beta e^{rh}$ 
\\
Entonces, para que el portafolio replique se tiene que cumplir 
$$\Bigg \{_{\alpha(F_0(d)-F_0)+\beta e^{rh}=V_d}^{\alpha(F_0(u)-F_0)+\beta e^{rh}=V_u}$$
Hay que resolver este sistema con respecto a $\alpha$ y $\beta$ 
$$\alpha=\frac{\begin{matrix} |V_u & e^{rh}|\\ |V_d & e^{rh}| \end{matrix}}{\begin{matrix}| F_0(u-1) & e^{rh}|\\| F_0(d-1) & e^{rh}|  \end{matrix}}$$
$$=\frac{V_u-V_d}{F_0(u-d)}$$
Para $\beta$: 
$$\beta=\frac{\begin{matrix} |F_0(u)-F_0 & V_u|\\ |F_0(d)-F_0 & V_d| \end{matrix}}{e^{rh}F_0[(u-1)-(d-1)]}=\frac{F_0(u-1)V_d-F_0(d-1)V_u}{e^{rh}F_0[(u-1)(d-1)]}$$
$$=e^{-rh}\frac{V_d(u-1)-V_u(d-1)}{u-d}$$
Entonces, 
$$V_o=\alpha (0)+\beta =\beta=e^{-rh}\frac{V_d(u-1)-V_u(d-1)}{u-d}$$
$$\Rightarrow V_0=e^{-rh}[\frac{1-d}{u-d}V_u + \frac{u-1}{u-d}V_d]$$
$$=e^{-rh}[p^{*}V_u +(1-p^{*})V_d]$$
Con, $$ p^{*}:= \frac{1-d}{u-d} \, \, \, 1-p^{*}:= \frac{u-1}{u-d} $$

$$\therefore V_o=e^{-rh}[\frac{1-d}{u-d}V_u + \frac{u-1}{u-d}V_d]=e^{-rh}[p^{*}V_u +(1-p^{*})V_d]$$ 

\section{Formula de B\&S para calls y puts sobre futuros}
$$C(K,T)=F_0e^{-rT}\Phi (d_1)-Ke^{-rT} \Phi (d_2)$$
$$P(K,T)=Ke^{-rT}\Phi (-d_2)-F_0e^{-rT} \Phi (-d_1)$$
Con, $d_1:=\frac{log(\frac{F_0}{K})+\frac{1}{2} \sigma^2 T}{\sigma\sqrt{T}} \, \, \,\,,  d_2:=d_1-\sigma \sqrt{T}=\frac{log(\frac{F_0}{K})-\frac{1}{2} \sigma^2 T}{\sigma\sqrt{T}}$
\\ 
\\
\textbf{Observación:}\\ 
$$C(F_0,K,T)-P(F_0,K,T)=F_0 e^{-rT}(\Phi (d_1)+\Phi (d_2))-Ke^{-rT}(\Phi (d_2)+\Phi (-d_2))$$ 
$$=F_0e^{-rT}-Ke^{-rT}$$
$$=e^{-rT}(F_0-K)+0$$
Que es lo que esperabamos. 






\end{document}