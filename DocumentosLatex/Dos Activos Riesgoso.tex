\documentclass[12pts]{extarticle}
\usepackage[utf8]{inputenc}
\usepackage[margin = 1in]{geometry}
\usepackage[spanish]{babel} 
\decimalpoint
\usepackage{amsmath}
\usepackage{mathtools}
\usepackage{amsfonts} % mathbb
\usepackage{mathrsfs} % mathscr
\usepackage{enumitem} % letras en enumerate
\usepackage[bottom]{footmisc}
\usepackage{amssymb}
\usepackage{float}
\usepackage{graphicx}
\graphicspath{{imag/}}
\setlength\parindent{0pt} % Quita indents de todo el documento
\author{}
\date{}
\begin{document} 
\section{Dos activos riesgosos}
Aunque en clase ya se ha dado una expresión para la ponderación del portafolio de mínima varianza, se dará una expresión para $n=2$, que \textbf{no} utiliza la elaborada notación matricial que se ha usado. \\ 
Como antes, se considerará un portafolio con 2 activos riesgosos, es decir,  $r=\alpha r_1 +(1-\alpha)r_2$. \\
 Ya se vío anteriormente que $Var(r)=\alpha^2\sigma_1^2 + (1-\alpha)^2 \sigma_2 ^2 +2\alpha (1-\alpha)\rho \sigma_1 \sigma_2$. \\
 Entonces para encontrar el portafolio de mínima varianza tendremos que resolver lo siguiente: 
$$\frac{\delta}{\delta\alpha} Var(r)= \frac{\delta}{\delta\alpha} (\alpha^2 \sigma_1^2 + (1-\alpha)^2 \sigma_2^2 +2\alpha (1-\alpha)\rho\sigma_1\sigma_2)$$ 
 $$=2\alpha\sigma_1^2 +2(1-\alpha)(-1)\sigma_2^2 + 2\rho\sigma_1\sigma_2(1-2\alpha)$$ $$=2[\alpha\sigma_1^2 -(1-\alpha)\sigma_2^2 + \rho\sigma_1\sigma_2(1-2\alpha)]$$ $$=2[\alpha(\sigma_1^2 +\sigma_2^2 -2\rho\sigma_1\sigma_2) -\sigma_2^2 +\rho\sigma_1\sigma_2]$$
\\ De aquí que $$\frac{\delta}{\delta\alpha} Var(r)=0$$ \\ si y solo sí $$\alpha=\frac{\sigma_2^2 -\rho \sigma_1 \sigma_2}{\sigma_1^2  \sigma_2^2 -2\rho \sigma_1 \sigma_2} = \frac{\sigma_2^2 -\rho\sigma_1\sigma_2}{Var(r_1 - r_2)}$$
\\ \textbf{ Nótese que $\frac{\delta^2}{\delta\alpha^2} Var(r) = 2(\sigma_1^2 +\sigma_2^2 -2\rho\sigma_1\sigma_2) = 2Var(r_1 - r_2) \geq 0$. Por lo que dicho $\alpha$ es mínimo.} 
\\ 
\section{$n$ activos riesgosos y uno libre de riesgo} 
\begin{itemize}
\item Un portafolio puede ser la combinación de cualquier portafolio en la bala (de activos puramente riesgosos) con el activo libre de riesgo.
\item En este framework de combinación, la frontera eficiente esta formada por los pórtafolios que están en la línea recta que satisface:  
\begin{enumerate}
\item Pasa por el punto libre de riesgo $(0,r_f)$
\item Es tangencial a la frontera eficiente formada sólo por los activos puramente riesgosos. 
\end{enumerate}
\begin{center}
\includegraphics[scale=.5]{CML}
\end{center}
\item La pendiente de la CML es $\frac{\mathbb{E}(r_m) - r_f}{SD(r_m)} = SR(r_m)$
\item Cualquier inversionista racional seleccionará un portafolio en la CML. El punto exacto dependerá de la preferencia de riesgo del inversionista. 
\item Cada punto en la CML es una combinación del activo libre de riesgo y M (el portafolio puramente riesgoso con el Sharpe-Ratio más alto)
\item Todo inversionista racional posee M y se conoce como \textbf{portafolio riesgo optimo} o \textbf{portafolio de tangencia}
\item El hecho de que la selección de la combinación de activos riesgosos para formar el potafolio riesgoso optimo sea independiente de la preferencia de riesgo se conoce como \textbf{Mutual Fund Separation Theorem} 
\end{itemize}
Ahora, para obtener el portafolio riesgoso optimo necesitamos encontrar las ponderaciones del portafolio tal que el logaritmo del Sharpe-Ratio se máximice. 
$$ r_M =\alpha_1 r_1 + \alpha_2 r_2,  \alpha_1 + \alpha_2=1$$ 
Entonces $$\mathbb{E}(r_M) - r_f = \alpha_1 \mathbb{E}(r_1) + \alpha_2 \mathbb{E}(r_2) -r_f = \alpha_1 [\mathbb{E}(r_1) -r_f] +\alpha_2[\mathbb{E}(r_2)-r_f]$$
Y, $$Var(r_M)=\alpha_1 \sigma_1^2 + \alpha_2^2 \sigma_2^2 + 2\rho\sigma_1\sigma_2\alpha_1\alpha_2$$
De esta forma, tenemos que el Sharpe-Ratio es: 
$$ SR(r_M)= \frac{\alpha_1 [\mathbb{E}(r_1) -r_f]+\alpha_2 [\mathbb{E}(r_2) -r_f]}{(\alpha_1^2 \sigma_1^2 + \alpha_2^2 \sigma_2^2 + 2\rho\sigma_1\sigma_2\alpha_1\alpha_2)^\frac{1}{2}}$$
Y el logaritmo del Sharpe-Ratio es: $$ h(\alpha_1 , \alpha_2) := log(\mathbb{E}(r_M)-r_f)-\frac{1}{2} log(Var(r_M))$$ 
$$=log(\alpha_1 [\mathbb{E}(r_1)- r_f]+ \alpha_2 [\mathbb{E}(r_2) -r_f])- \frac{1}{2} log(\alpha_1^2  \sigma_1^2 + \alpha_2^2 \sigma_2^2 +2 \rho \sigma_1 \sigma_2 \alpha_1 \alpha_2)$$
\textbf{Proposición}: Para cualquier $c>0$, y para cualesquiera $\alpha_1,\alpha_2 \in \mathbb{R},     h(c\alpha_1, c\alpha_2)= h(\alpha_1, \alpha_2)$
\\ \begin{itemize}
 \item En  virtud de esta proposición, si $(\alpha_1, \alpha_2)$ es un maximo de $h(*,*)$, entonces $(c\alpha_1, c\alpha_2)$ también es un máximo de $h(*,*)$. 
\item Esto significa que si se puede encontrar una solución $(\alpha_1, \alpha_2)$ que maximice $h(*,*)$, dicha solución se puede reescalar para que se satisfaga la restriccón $\alpha_1 + \alpha_2 =1$
\end{itemize} 
Se puede maximizar $h(*,*)$ con técnicas de cálculo tradicionales:
$$\frac{\delta}{\delta \alpha_1} h(\alpha_1, \alpha_2) = \frac{\mathbb{E}(r_1) -r_f}{\alpha_1[\mathbb{E}(r_1)-r_f]+\alpha_2[\mathbb{E}(r_2)-r_f]} - \frac{1}{2} \frac{2\alpha_1\sigma_1^2 + 2\alpha_2 \rho\sigma_1\sigma_2}{\alpha_1^2  \sigma_1^2 +\alpha_2^2 \sigma_2^2 +2\alpha_1\alpha_2\rho\sigma_1\sigma_2}$$  $$=\frac{\mathbb{E}(r_1)-r_f}{\mathbb{E}(r_M)-r_f} -\frac{\alpha_1\sigma_1^2 +\alpha_2\rho\sigma_1\sigma_2}{Var(r_M)}$$
Análogamente, $$\frac{\delta}{\delta\alpha_2} h(\alpha_1,\alpha_2)= \frac{\mathbb{E}(r_2)-r_f}{\mathbb{E}(r_M)-r_f} - \frac{\alpha_1\rho\sigma_1\sigma_2 + \alpha_2\sigma_2^2}{Var(r_M)}$$
Entonces, para que $\alpha_1, \alpha_2$ sean máximos, necesitamos $$\bigg\{_{\frac{\delta h}{\delta\alpha_2}=0 ...(II)}  ^{\frac{\delta}{\delta\alpha_2}=0 ...(I)}$$
Para la ecuación $(I)$, 
$$ \frac{\mathbb{E}(r_1) -r_f}{\mathbb{E}(r_M)-r_f}=\frac{\alpha_1 Cov(r_1, r_1) +\alpha_2 Cov(r_1, r_2)}{Var(r_M)} = \frac{Cov(r_1, \alpha_1 r_1 +\alpha_2 r_2)}{Var(r_M)} = \frac{Cov(r_1, r_M)}{Var(r_M)}$$
Es decir, $$\frac{\mathbb{E}(r_1)-r_f}{Cov(r_1, r_M)}=\frac{\mathbb{E}(r_M)-r_f}{Var(r_M)} ... (I*)$$
Análogamente, para la ecuación $(II)$,  $$\frac{\mathbb{E}(r_2) -r_f}{Cov(r_2, r_M)}=\frac{\mathbb{E}(r_M) -r_f}{Var(r_M)} ..(II*)$$ 
De $(I*)$ y $(II*)$ observese que: $$\frac{\mathbb{E}(r_1)-r_f}{Cov(r_1, r_M)}=\frac{\mathbb{E}(r_M)-r_f}{Var(r_M)}=\frac{\mathbb{E}(r_2) -r_f}{Cov(r_2, r_M)}$$
¿Este arugumento se puede urilizar para $n$ activos riesgosos? 
\\  \\ Notemos que en la expresión $\frac{\mathbb{E}(r_i) -r_f}{Cov(r_i, r_M)}$ sólo $Cov(r_i, r_M)$ depende de las ponderaciones del portafolio, entonces para obtener las ponderaciones del portafolio optimo: 

\begin{enumerate} 
\item Se encuentran las ponderaciones (que no necesariamente suman 1) tales que \\$Cov(r_i, r_M)= \mathbb{E}(r_i)-r_f$, para  $i= \{1,2,...,n\} $, es decir $\frac{\mathbb{E}(r_i)-r_f}{Cov(r_i,r_M)}=cte=1$
\item Se reescalan las ponderaciones para que sumen 1. 
\end{enumerate}.
Veamos el caso cuando $n=2$: 
$$\bigg\{_{Cov(r_2, r_M)=\mathbb{E}(r_2)-r_f} ^{Cov(r_1, r_M)= \mathbb{E}(r_1)-r_f} \rightarrow \bigg\{_{Cov(r_2, \alpha_1 r_1 +\alpha_2 r_2)=\mathbb{E}(r_2)-r_f } ^{Cov(r_1, \alpha_1 r_1 +\alpha_2 r_2)=\mathbb{E}(r_1)-r_f}$$ 
$$\rightarrow \bigg\{_{\alpha_1 \rho \sigma_1 \sigma_2 +\alpha_2\sigma_2^2 =\mathbb{E}(r_2)-r_f} ^{\alpha_1 \sigma_1^2 +\alpha_1\rho\sigma_1\sigma_2 =\mathbb{E}(r_1)-r_f}$$
Nos da un sistema de ecuaciones de $2x2$. Resolviendo este sistema por el método de Cramer tenemos lo siguiente:
 $$ \alpha_1=\frac{\left| \begin{array}{cc}
                                    \mathbb{E}(r_1)-r_f & \rho\sigma_1\sigma_2 \\
                                    \mathbb{E}(r_2)-r_f & \sigma_2^2
                                          \end{array} \right|}{\left| \begin{array}{cc}
                                                                                     \sigma_1^2 & \rho\sigma_1\sigma_2\\
                                                                                     \rho\sigma_1\sigma_2 & \sigma_2^2
                                                                                     \end{array} \right|} =\frac{\sigma_2^2(\mathbb{E}(r_1)-r_f)-\rho\sigma_1\sigma_2( \mathbb{E}(r_2)-r_f)}{\sigma_1^2\sigma_2^1 -\rho^2\sigma_1^2\sigma_2^2} $$
y, 
$$\alpha_2=\frac{\left| \begin{array}{cc}
                                        \sigma_1^2 & \mathbb{E}(r_1)-r_f\\
                                        \rho\sigma_1\sigma_2 & \mathbb{E}(r_2)-r_f
                                        \end{array} \right|}{\sigma_1^2\sigma_1^2 (1-\rho)} = \frac{\sigma_1^2 (\mathbb{E}(r_2)-r_f)-\rho\sigma_1\sigma_2(\mathbb{E}(r_1)-r_f)}{\sigma_1^2\sigma_2^2(1-\rho)}$$

Y posteriormente se escalan los valores de $\alpha_1$ y $\alpha_2 $, dando como resultado:
$$ \frac{\alpha_1}{\alpha_1 +\alpha_2} \rightarrow \alpha_1, \frac{\alpha_2}{\alpha_1 + \alpha_2} \rightarrow \alpha_2 $$ 

\textbf{Ejemplo.} Las acciones 1 y 2 tienen las siguientes caracteristicas:
 \begin{center}
\begin{tabular}{ c c c }
 Acción  & Rendimiento medio & Volatilidad \\ 
 1 & 9\% & 30\% \\  
 2 & 15\%  & 50\%    
\end{tabular}
\end{center}
La correlación entre los rendimientos de las acciones es de $60\%$, la tasa de interés libre de riesgo es del $3\%$. Encontrar el \textbf{portafolio riesgoso optimo.} 


\end{document}
