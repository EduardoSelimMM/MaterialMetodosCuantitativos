\documentclass[12pts]{extarticle}
\usepackage[utf8]{inputenc}
\usepackage[margin = 1in]{geometry}
\usepackage[spanish]{babel} 
\decimalpoint
\usepackage{amsmath}
\usepackage{mathtools}
\usepackage{amsfonts} % mathbb
\usepackage{mathrsfs} % mathscr
\usepackage{enumitem} % letras en enumerate
\usepackage[bottom]{footmisc}
\usepackage{amssymb}
\usepackage{float}
\usepackage{graphicx}
\graphicspath{{imag/}}
\setlength\parindent{0pt} % Quita indents de todo el documento
\author{}
\date{}
\title{Browniano}
\begin{document} 
\maketitle
\textbf{Definiión:} Se dice que un proceso estocastico $(B_t)_{t\geq0}$ es un movimiento Browniano estandar si:
\begin{enumerate}
\item $B_0 =0$ c.s
\item $B_{t+s}|B_t =b \sim N(b,s)$
\item Tiene incrementos independientes, i.e, $B_{t+s_{1}} -B_t \perp B_t-B_{t-{s_2}}$ para cualesquiera $t, s_1, s_2 \in \mathbb{R}_+$
\end{enumerate}

\textbf{Observación:} Se puede demostrar que para casi todo $w \in \Omega$, la aplicación $t \mapsto B_t (w)$ es continua
\\ 
\textbf{Ejemplo:} El precio de una acción sigue un movimiento Browniano. El precio de la acción al tiempo 3 es de \$52. Calcule la probabilidad de que el precio de la acción sea de al menos \$55 al tiempo 12. 
\\ \\ \\ \\ \\ \\ \\ \\ \\ \\ 
¿Hace sentido que el movimiento Browniano sea un modelo para precios de activos? \\ NO, por muchas razones, la principal es que $B_t$ puede tomar cualquier valor real, y los precios en principio, son no negativos. 

\section{Movimiento Browniano Aritmetico} 
\textbf{Definición:} Un movimiento Browniano aritmetico $(X_t)_{t \geq 0}$ es un proceso estocastico que se puede escribir como $$ X_t =X_0+\alpha t +\sigma B_t$$ Donde $(B_t)_{t\geq 0}$ es un movimiento Browniano estandar. 
\\ Nótese que:
\begin{itemize}
\item $$X_{t+s}-X_t = (X_0 +\alpha (t+s)+\sigma B_{t+s})+(X_0+\alpha t +\sigma B_t)$$
$$=\alpha s +\sigma (B_{t+s} - B_t)$$  
\item $$\mathbb{E}(X_{t+s}-X_s)=\alpha s+\sigma \mathbb{E}(B_{t+s} - B_t)$$
         $$ =\alpha s +\sigma 0 =  \alpha s$$
\item $$ Var(X_{t+s}-X_s)=Var( \alpha s + \sigma (B_{t+s} - B_t)$$
         $$ =Var(\sigma(B_{t+s} - B_t))$$
         $$ = \sigma^2 Var(B_{t+s} - B_t)$$
         $$ =\sigma^2 s$$
\end{itemize}
Entonces, $$ X_{t+s}-X_t \sim N(\alpha s, \sigma^2 s)$$
Al parametro $\alpha$ se le conoce como \textbf{drift} del proceso. \\
Tambien nótese que: $$ X_{t+s}|X_t = x \sim N(x+\alpha s, \sigma^2 s)$$
\textbf{Ejemplo:} El precio de un acción sigue un movimiento Browniano aritmetico de la forma $X_t = X_0 +t+0.2B_t$. Si el precio actual de la acción es de \$40, calcule la probabilidad de que el precio de la acción al tiempo 4 sea menor de \$43. 
\\ \\ \\ \\ \\ \\ \\ \\ \\ \\ 
¿Hace sentido que el movimiento Browniano Aritmetico sea un modelo para precios de activos? \\
NO. Por lo mismo que un movimiento Browniano estandar. 
\\ 
\section{ Movimiento Browniano Geométrico}
\textbf{Definición:} Se dice que el proceso $(X_t)_{t\geq 0}$ es un movimiento Browniano geométrico si $(log(X_t))_{t\geq 0}$ es un movimiento Browniano aritmético. 
\\ Si, $$ log(\frac{X_t}{X_0}) \sim N(\mu t, \sigma^2 t)$$
Entonces, $$X_t \sim logN(\mu t, \sigma^2 t)$$
Y por lo tanto, $$\mathbb{E}(\frac{X_t}{X_0})=e^{ \mu t +\frac{1}{2}\sigma^2 t}$$
                         $$ Var(\frac{X_t}{X_0})=e^{2\mu t +\sigma^2 t} (e^{ \sigma^2 t}-1)$$
Pero también se puede escribir $log(\frac{X_t}{X_0})=\mu t +\sigma \sqrt{t} Z$, donde $Z\sim N(0,1)$
\\ $\Rightarrow \frac{X_t}{X_0} = exp\{\mu t+\sigma \sqrt{t} Z\}$
\\ $\Rightarrow X_t = X_0e^{\mu t+\sigma \sqrt{t} Z}$
\\ \\  El movimiento Browniano geometrico es un modelo popular para modelar precios de acciones. 
\begin{itemize} 
\item Si $(S_t)_{t \geq 0}$ sigue un movimiento Browniano geometrico, entonces $\frac{S_{t+n}}{S_0}$ se distribuye log-normal. 
\item Tambien $\frac{S_{t+n}}{S_t}$ se distribuye log-normal, cuyos parámetros no dependen de t. 
\end{itemize}

*\textbf{Disclaimer:} A partir de este momento, seré muy muy muy informal en la manera que escribo matematicas. Me interesan las ideas que la formalidad. 
\\
En cálculo clásico se sabe que si $y(t)=e^{rt}$, entonces $$\frac{dy}{dt}=ce^{ct}$$
Escrito en forma de "diferenciales"
$$dy=ce^{rt}dt=cydt$$
Es decir, $\frac{dy}{y}=cdt$ (que ya se parecea alguna de las ecuaciones diferenciales que conocemos) \\
Supongase que hay cierta incertidumbre en esta tasa de cambio, i.e., se escribirá $\frac{dy}{y}=cdt+\sigma dB_t$, donde $\sigma dB_t$ es el "término de error".
\\
En este momento se puede decir que $dB_t$ es la "diferencial" de un movimiento \textbf{Browniano Estandar}, que se puede pensaer como el límite ciando $n\rightarrow 0$, de una variabe aleatoria igual a $h$ con probabilidad $\frac{1}{2}$ e igual a $-h$ con probabilidad $\frac{1}{2}$ \\\\
\textbf{Ejemplo:} En un movimiento Browniano aritmetico $X_t=X_0+\mu t+\sigma B_t$ se escribe la "diferencial"de éste como $$dX_t=\mu dt+\sigma dB_t$$
"Un pequeño cambio en X es igual a $\mu$ veces un pequeño cambio en el tiempo más $\sigma$ veces un pequeño cambio en el movimiento Browniano" 
\\ 
\textbf{Ejemplo:} En un movimiento Browniano Estandar $(B_t)_t\geq0$, $B_t \sim N(0, t)$, por lo tanto $dB_t\sim N(0, dt)$ \\
El valor al tiempo t de un movimiento Browniano geometrico $(X_t)t\geq0$ se puede expresar en términos de su logaritmo 
$$log(X_t)=log(X_0)+(\zeta -\frac{1}{2}\sigma^2)t + \sigma B_t$$ 
La "diferencial" de esta expresión es 
$$ d(log(X_t))=(\zeta -\frac{1}{2}\sigma^2)dt+\sigma dB_t$$ 
Y se dijo que este movimiento Browniano geometrico se puede expresar como $$ X_t=X_0 exp{(\zeta -\frac{1}{2}\sigma^2)t+\sigma B_t}$$
Más adelante se probará que la dferencial de este movimiento Browniano es $$dX_t=\zeta X_t dt + \sigma X_t dB_t$$
$$\Rightarrow \frac{dX_t}{X_t}=\zeta dt +\sigma dB_t$$ 
\textbf{Ejemplo:} Supongase que el proceso de precios de una acción sigue la dinamica $dS_t=0.25S_t dt+0.10S_tdB_t$, calcular la probabildad de que $S_t$ sea al menos $5\%$ más grande que $S_0$ en: 
\begin{itemize}
\item t=1
\item t=0.1
\end{itemize}

\textbf{Solución:}
La dinamica de la acción es un movimiento Browniano geometrico con $\zeta=0.25$ y $\sigma=0.10$, por lo tanto $\mu=\zeta-\frac{1}{2}\sigma^2$ es el parametro correspondiente al movimiento Browniano Arirmetico, entonces $$d(log(S_t))=(0.25-\frac{1}{2}(0.1)^2)dt+0.1dB_t$$ $$=0.245dt+0.1dB_t$$
Entonces $$\mathbb{P}(\frac{S_t}{S_0}\geq 1.05)=\mathbb{P}(log(S_t)-log(S_0)\geq log(1.05))$$
\begin{itemize}
\item Para $t=1$, $m=0.245$ y $v=0.1$
$$\mathbb{P}(log(S_t)-log(S_0)\geq log(1.05))$$
$$=\mathbb{P}(\frac{log(S_t)-log(S_0)-0.245}{0.1}\geq \frac{log(1.05)-0.245}{0.1})$$
$$=1-\Phi(\frac{log(1.05)-0.245}{0.1})=1-\Phi(-1.96210)$$
$$=\Phi(1.96210)=0.97512=\mathbb{P}(S_1\geq S_0(1.05))$$ 
\item Para $t=0.1$, $m=0.245(0.1)$, $v=0.1\sqrt{0.1}$ 
$$\mathbb{P}(log(S_t)-log(S_0)\geq Log(1.05))$$
$$=\mathbb{P}(\frac{log(S_t)-log(S_0)-0.0245}{0.1\sqrt{0.1}} \geq \frac{log(1.05)-0.0245}{0.1\sqrt{0.1}}$$
$$=1-\Phi(0.26812)=0.22121=\mathbb{P}(S_{0.1} > S_0 (1.05))$$
\end{itemize} 
\textbf{Ejemplo:} Se sabe que $(S_t)_{t>0}$ sigue una dinamica estocastisca definida mediante $\frac{dS_t}{S_t}=0.15dt+0.2dB_t$, dado que $S_0=40$, calcular la probabilidad de que $S_{13} \in (40,50)$
\\ 
\textbf{Solucion:}   \\
$(S_t)_{t>0}$ sigue un movimiento Browniano geometrico. $$ \mathbb{P}(10<S_13 <50 | S_q =40)=\mathbb{P}(40<S_{13-9}<50 | S_{9-9}=40)$$  $$\mathbb{P}(40<S_4 <50 | S_0=40)$$
Los parámetros asociados a la variable aleatoria normal son: $$\mu t = 4(0.15-\frac{1}{2}(0.2)^2)=0.52$$  $$\sigma \sqrt{t} =0.2 \sqrt{4} =0.4$$
Así que, $$\mathbb{P}(\frac{40}{40} < \frac{S_4}{S_0} <\frac{50}{40}) = \mathbb{P}(log(1)<log(\frac{S_4}{S_0}) < log(\frac{5}{4}) $$ $$=\mathbb{P}(\frac{log(1)-0.52}{0.4} \leq \frac{log(\frac{S_4}{S_0})-0.52}{0.4} \leq \frac{log(\frac{5}{4})-0.52}{0.4})$$
$$ \Phi (\frac{log(\frac{5}{4})-0.52}{0.4})-\Phi(\frac{-0.52}{0.4})$$ $$\Phi(-0.74214)-\Phi(-13)$$ $$=0.1322$$ 

Formalmente, cuando se escribe la expresión $$dX_1 =a(t, X_t)+b(t, X_t)dB_t$$ Se esta haciendo referencia a la siguiente ecuación $$X_t = X_0 + \int_{0}^{t} a(u,X_u) \,du + \int_{0}^{t} b(u,X_u) \,dB_u$$

\textbf{El lema de Ito} \\
 
\textbf{Definición:} Un proceso de Ito es un proceso estocástico $(X_t)_{0 \leq t}$ cuya diferencial se puede expresar como: $$ dX_t = a(t,X_t)dt +b(t,X_t)dB_t  $$
\begin{itemize}
\item El movimiento Browniano aritmetico es un proceso de Ito con $a(t,X_t)=\alpha$ y $b(t,X_t)=\sigma X_t$ 
\item El movimiento Browniano geometrico es un proceso de Ito, con $a(t, X_t)=cX_t$ y $ b(t, X_t)=\sigma X_t$ 
\end{itemize}
 
Un ejemplo importante es: $$dX_t =\lambda (\alpha -X_t)dt +\sigma dB_t $$
i.e., $a(t, X_t)=\lambda (\alpha-X_t)$ y $b(t,X_t)=\sigma$. \\ \\
"La" solución a esta ecuaciónse conoce como proceso de \textbf{Ornstein-Uhlenbeck} \\
\begin{itemize}
\item A $a(t,X_t)$ se le conoce como funcion de \textbf{drift} y a $b(t,X_t)$ se le conoce como función de volatilidad o función de difusión. 
\item El lema de Ito es una fórmula para "calcular" $dg(S,t)$, i.e., la "diferencial" de una función S y t donde S es estocastico. 
\end{itemize}
\textbf{Proposición (lema de Ito)}: 
$$dg=\frac{\delta g}{\delta S} ds + \frac{\delta g}{\delta t} dt +\frac{1}{2} \frac{\delta^2 g}{\delta S^2} (dS)^2 $$
Donde para obtener $(dS)^2$ se utiliza la siguiente tabla: 
\begin{center}
 \begin{tabular}{||c c c ||} 
 \hline
  & dt & $dB_t$\\ [0.5ex] 
 \hline\hline
 dt & 0 & 0 \\ 
 \hline
$dB_t$ & 0 & dt \\
 \hline
\end{tabular}
\end{center}

\textbf{Observaciones:} Según esta tabla de "multiplicación" 
\begin{itemize}
\item $dB_t(dB_t)=dt$ 
\item $(dB_t)^3 = dB_t(dB_t)(dB_t)=dt(dB_t)=0$ 
\item $(dB_t)^4=(dB_t)^3(dB_t)=0(dB_t)=0$
\end{itemize}

\textbf{Ejemplo:} \\
Si $X_t=\alpha t +  \sigma B_t $, calcular $dX_t$ usando el lema de Ito. \\
\textbf{Solución:} \\
$X_t$ es una función de $B_t$, por tanto X juega el papel de g y B juega el papel de S en el lema de Ito. 

\begin{itemize}
\item $\frac{\delta X}{\delta B}=\frac{\delta}{\delta B} (\alpha t +\sigma B) =\sigma $
\item $\frac{\delta^2 X}{\delta B^2}=\frac{\delta}{\delta B}\sigma =0$
\item $\frac{\delta X}{\delta t}=\frac{\delta}{\delta t}(\alpha t+\sigma B)=\alpha $
\end{itemize}
Etronces, según el lema de Ito, $$ dX_t=\frac{\delta X}{\delta B} dB_t + \frac{\delta X}{\delta t} dt + \frac{1}{2} \frac{\delta^2 X}{\delta B^2} (dB)^2 $$
$$= \sigma dB_t 6 \alpha dt + \frac{1}{2} 0 dt $$ 
$$=\alpha dt + \sigma dB_t$$

\textbf{Ejemplo:} Suponga que $X_t=X_0 exp{(c-\frac{1}{2}\sigma^2)t +\sigma B_t}$, calcular $sX_t$ usando el lema de Ito. 
\\
$$\frac{\delta X}{\delta B}= \frac{\delta}{\delta B} (X_0 exp{(c-\frac{1}{2}\sigma^2)t +\sigma B})$$ $$=X_0 exp{(c*-\frac{1}{2}\sigma^2)t +\sigma B}\sigma$$ $$=\sigma X$$ 

$$\frac{\delta^2 X}{\delta B^2}= \frac{\delta}{\delta B}(\sigma X)$$ $$=\sigma \frac{\delta X}{\delta B}$$ $$ =\sigma (\sigma X)=\sigma^2 X$$
$$\frac{\delta X}{\delta t} =\frac{\delta}{\delta t} (X_0 exp{(c-\frac{1}{2}\sigma^2)t +\sigma B})$$ $$=X_0 exp{(c-\frac{1}{2}\sigma^2)t +\sigma B_t})(c-\frac{1}{2})$$ $$=(c-\frac{1}{2}\sigma^2)X$$
Entonces, por el lema de Ito
 $$ dX_t=\frac{\delta X}{\delta B} dB_t + \frac{\delta X}{\delta t} dt + \frac{1}{2} \frac{\delta^2 X}{\delta B^2} (dB)^2 $$
$$=\sigma X_t dB_t + (c-\frac{1}{2}\sigma^2)X_t dt+ \frac{1}{2}\sigma^2 X_t (dB_t)^2$$
$$=\sigma X_t dB_t + (c-\frac{1}{2}\sigma^2)X_t dt+ \frac{1}{2}\sigma^2 X_t dt$$
$$=\sigma X_t dB_t + cX_t dt$$
$$=cX_t dt +\sigma X dB_t$$
$$\Rightarrow \frac{dX_t}{X_t}=c dt + \sigma dB_t$$

\textbf{Ejemplo:} Se sabe que $\frac{d X_t}{X_t}=\zeta dt + \sigma dB_t$. Calcular $d(log(X_t))$ utilizando el lema de Ito.

\textbf{Solución:} \\
Sea $Y_t := log(X_t)$. \\
$$\Rightarrow  \frac{\delta y}{\delta X} = \frac{1}{X} $$ 
$$ \frac{\delta^2 Y}{\delta X^2}=-\frac{1}{X^2} $$
$$\frac{\delta Y}{\delta t}=0$$
Entonces, $$dY_t =\frac{1}{X_t}dX_t+0dt+\frac{1}{2}(-\frac{1}{X^2})(dX_t)^2$$
$$=\zeta dt +\sigma dBt +\frac{1}{2}(-\frac{1}{X_t^2}(dX_t)^2$$
$$=\zeta dt+ \sigma dB_t -\frac{1}{2}(\frac{dX_t}{X_t})^2$$
$$=\zeta dt +\sigma dBt - \frac{1}{2}[\zeta^2 dtdt + \sigma^2 dB_t dB_t+2\zeta \sigma dtBt]$$
$$=\zeta dt+\sigma dB_t-\frac{1}{2}\sigma^2 dt$$

\textbf{Ejemplo:} Se sabe que el precio de una acción sigue un proceso de Ito $dSt=\zeta S_t dt +\sigma S_t dB_t$. Considere un derivado sobre S que al tiempo t paga $S_t^5$ i.e., $C(S_t, t)=S_t^5$. Demuestre que $(Log(C(S_t, t)))_{0 \leq t}$ es un proceso de Ito. 

\textbf{Solución:} 
$$\frac{\delta C}{\delta S}=\frac{\delta}{\delta S}(S^5)=5S^4$$
$$\frac{\delta^2 C}{\delta S^2}=\frac{\delta}{\delta S}(5S^4)=20S^3$$
$$\frac{\delta C}{\delta t}=\frac{\delta}{\delta t}(S^5)=0$$
Entonces, por el lema de Ito:
$$dC=5S_t^4 dS_t +0dB_t +\frac{1}{2} 20S_t^3 (dS_t)^2$$
$$=5S_t^4 ds + \frac{1}{2}20S_t^3(dS_t)^2 ... (*)$$
Pero, $$(dS_t)^2=(\zeta S_t dt +\sigma S_t dB_t)^2$$
$$=\zeta^2 S_t^2 dt dt + 2\zeta \sigma S_t^2 dtB_t + \sigma ^2 S_t^2 dB_t dB_t$$
$$=\zeta^2 S_t^2 (0)+2\zeta \sigma S_t^2 (0) + \sigma^2 S_t^2 dt $$
$$=\sigma^2 S_t^2 dt$$
Así, sustituyendo en (*) tenemos que:
$$dC=5S_t^4 dS_t +10 S_t^3 \sigma^2 S_t^2 dt $$
$$=5S_t^4 (\zeta S_t dt + \sigma S_t dB_t)+10\sigma^2 S_t^5 dt $$
$$=5\zeta S_t^5 +5\sigma S_t^5 dB_t +10 \sigma^2 S_t^2 dt$$
$$=(5\zeta S_t^5 +10  \sigma S_t5)dt +5\sigma S_t^5 dB_t$$
$$=S_t^5[(5 \zeta +10 \sigma^2)dt +5\sigma dB_t]$$
$$=C(S,t)[(5\zeta +10 \sigma^2) dt + 5\sigma dB_t]$$ 

Por el ejemplo 3, $$d(log(C_t))=(5\zeta +10 \sigma^2-\frac{1}{2}(5\sigma)^2)dt +(5\sigma)dB_t$$ $$=(5\zeta - \frac{5}{2}\sigma^2)dt + 5\sigma dB_t$$
Haciendo $(5\zeta - \frac{5}{2}\sigma^2) = a$ y $ 5\sigma = b$
$$\Rightarrow dlog(C_t)=adt  +b dt $$

$\therefore (log(C_t))_{0 \leq t}$ es un proceso de Ito.  \\ \\

$\rightarrow$ \textbf{Más adelante veremos más ejemplos} 


 



























\end{document}